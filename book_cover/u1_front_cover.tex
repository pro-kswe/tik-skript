% !TEX root = ../main.tex

\begingroup
\thispagestyle{empty}

\ifthenelse{\boolean{withsolutions}}
{
\begin{tikzpicture}[remember picture, overlay]
\node[inner sep=0pt] (background) at (current page.center) {\includegraphics[width=\paperwidth]{chip}};
\draw (current page.center) node [fill={ocre}, text opacity=1, inner sep=1cm]{\Huge\centering\bfseries\sffamily\parbox[c][][t]{\paperwidth}{\centering \textcolor{white}{\uppercase{Technische Informatik}}\\[14pt]
{\huge  \textcolor{white}{Digitaltechnik}}\\
{\small  \textcolor{white}{inkl. Lösungsvorschläge}}\\
{\Large  \textcolor{white}{Grundlagenfach Informatik}}}};
\end{tikzpicture}
}
{
\begin{tikzpicture}[remember picture, overlay]
\node[inner sep=0pt] (background) at (current page.center) {\includegraphics[width=\paperwidth]{chip}};
\draw (current page.center) node [fill={ocre}, text opacity=1, inner sep=1cm]{\Huge\centering\bfseries\sffamily\parbox[c][][t]{\paperwidth}{\centering \textcolor{white}{\uppercase{Technische Informatik}}\\[14pt]
{\huge  \textcolor{white}{Digitaltechnik}}\\[14pt]
{\Large  \textcolor{white}{Grundlagenfach Informatik}}}};
\end{tikzpicture}
} 

\vfill
\endgroup
