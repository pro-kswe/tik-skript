% !TEX root = ../../../main.tex

\toggletrue{image}
\togglefalse{imagehover}
\chapterimage{logic_boat_1}
\chapterimagetitle{\uppercase{Logic Boat 1}}
\chapterimageurl{https://xkcd.com/1134/}
\toggletrue{imagehover}

\chapter{Boolesche Algebra}
\label{chapter-boolesche-algebra}

Grundlage der Logikgatter ist die \textbf{boolesche Algebra}\footnote{Benannt nach Georg Boole, 1815 - 1864}. Für jedes Logikgatter gibt es ein \say{Rechensymbol}. Das Verhalten eines Logikgatters wird dann durch eine \textbf{boolesche Formel} dargestellt. Die Lernziele sind:\\

\newcommand{\boolescheAlgebraLernziele}{
\protect\begin{minipage}{\textwidth}
\begin{todolist}
\item Sie nennen für die drei Logikgatter \texttt{UND}, \texttt{ODER} und \texttt{NICHT} jeweils die boolesche Operation.
\item Sie notieren für jedes der drei Logikgatter den booleschen Operator.
\item Sie stellen das Verhalten für jedes der drei Logikgatter als boolesche Formel dar.
\end{todolist}
\end{minipage}
}

\lernziel{\autoref{chapter-boolesche-algebra}, \nameref{chapter-boolesche-algebra}}{\protect\boolescheAlgebraLernziele}

\boolescheAlgebraLernziele

\section{Konjunktion}

Die boolesche Operation für das \texttt{UND}-Gatter ist die \textbf{Konjunktion} der booleschen Algebra. Wir notieren die boolesche Formel für das Verhalten des \texttt{UND}-Gatters wie folgt:

\begin{center}
\Large
$\underbrace{A_0 = \overbrace{E_0 \wedge E_1}^{\text{Konjunktion}}}_{\text{Boolesche Formel}}$
\end{center}

Das $\wedge$-Symbol ist der boolesche Operator für die Konjunktion. Die Formel wird ausgesprochen als \say{$A_0$ gleich $E_0$ \textbf{und} $E_1$}.

\begin{hinweis}
Den Begriff Operator kennen Sie bereits aus der Mathematik. Bei der Addition ist $+$ ein (arithmetischer) Operator. Meist wird er auch einfach als Rechenzeichen bezeichnet.
\end{hinweis}

\section{Disjunktion}

Die boolesche Operation für das \texttt{ODER}-Gatter ist die \textbf{Disjunktion} der booleschen Algebra. Wir notieren die boolesche Formel für das Verhalten des \texttt{ODER}-Gatters wie folgt:

\begin{center}
\Large
$\underbrace{A_0 = \overbrace{E_0 \vee E_1}^{\text{Disjunktion}}}_{\text{Boolesche Formel}}$
\end{center}

Das $\vee$-Symbol ist der boolesche Operator für die Disjunktion. Die Formel wird ausgesprochen als \say{$A_0$ gleich $E_0$ \textbf{oder} $E_1$}.

\section{Negation}

Die boolesche Operation für das \texttt{NICHT}-Gatter ist die \textbf{Negation} der booleschen Algebra. Die boolesche Formel für das Verhalten des \texttt{NICHT}-Gatters wird wie folgt notiert:

\begin{center}
\Large
$\underbrace{A_0 = \overbrace{\neg E_0}^{\text{Negation}}}_{\text{Boolesche Formel}}$
\end{center}

Das $\neg$-Symbol ist der boolesche Operator für die Negation. Die Formel wird ausgesprochen als \say{$A_0$ gleich \textbf{nicht} $E_0$}.

\section{Übungen}

\begin{exercise}
Füllen Sie die nachstehende Tabelle aus. Verwenden Sie $E_0$, $E_1$ und $A_0$.

\begin{table}[htb]
\centering
\begin{tblr}{
colspec={|Q[c, m]|[2pt]Q[c, m, 3.5cm]|[2pt]Q[c, m, 3.5cm]|[2pt]Q[c, m, 3.5cm]|}
}
\hline
& \textbf{UND} & \textbf{ODER} & \textbf{NICHT} \\ \hline[2pt]
\textbf{Boolesche}\\ {\textbf{Formel}} &          &               &                \\ \hline
\textbf{Boolesche}\\ {\textbf{Operation}} &      &               &                \\ \hline
\textbf{Boolescher}\\ {\textbf{Operator}} &    &               &                \\ \hline
\textbf{Logikgatter}         &        \vspace{1.25cm}        &               &                \\ \hline
\end{tblr}
\end{table}
\end{exercise}
\begin{solution}
\begin{table}[H]
\centering
\begin{tblr}{
colspec={|Q[c, m]|[2pt]c|[2pt]c|[2pt]c|}
}
\hline
& \textbf{UND} & \textbf{ODER} & \textbf{NICHT} \\ \hline[2pt]
\textbf{\small Boolesche Formel} & $A_0 = E_0 \wedge E_1$ & $A_0 = E_0 \vee E_1$ & $A_0 = \neg E_0$ \\ \hline
\textbf{\small Boolesche Operation} & {Konjunktion \\ $E_0 \wedge E_1$} & {Disjunktion \\ $E_0 \vee E_1$} & {Negation \\ $\neg E_0$} \\ \hline
\textbf{\small Boolescher Operator} & $\wedge$ & $\vee$ & $\neg$ \\ \hline
\textbf{\small Logikgatter} &        
\begin{circuitikz}[baseline={(current bounding box.center)}]
\draw (0,0) node[and port] (AND1) {}
(AND1.in 1) node[anchor=east] {$E_0$} 
(AND1.in 2) node[anchor=east] {$E_1$}
(AND1.out) node[anchor=west] {$A_0$};
\end{circuitikz}
      &               
      \begin{circuitikz}[baseline={(current bounding box.center)}]
\draw (0,0) node[or port] (OR1) {}
(OR1.in 1) node[anchor=east] {$E_0$} 
(OR1.in 2) node[anchor=east] {$E_1$}
(OR1.out) node[anchor=west] {$A_0$};
\end{circuitikz}
      &
      \begin{circuitikz}[baseline={(current bounding box.center)}]
\draw (0,0) node[european not port] (NOT1) {}
(NOT1.in) node[anchor=east] {$E_0$} 
(NOT1.out) node[anchor=west] {$A_0$};
\end{circuitikz}
      \\ \hline
\end{tblr}
\end{table}
\end{solution}

\begin{exercise}
Welche Gesetze gelten in der booleschen Algebra? Übertragen Sie die booleschen Formeln in Logicly. Ergänzen Sie anschliessend die Gleichungen in der Tabelle.
\begin{itemize}
\item Eingänge ($E_0$ bzw. $E_1$) sind Schalter. Ausgänge ($A_0$) sind Glühlampen.
\item \num{0} bedeutet \say{Low Constant}, \num{1} bedeutet \say{Hight Constant}
\end{itemize}

\num{0} bedeutet, dass kein Strom fliesst. \num{1} bedeutet, dass Strom fliesst.

\begin{table}[htb]
\centering
\begin{tblr}{
colspec={|c|Q[c, l, 4cm]|Q[c, l, 4cm]|},
stretch=1.5
}
\hline
\textbf{Extremalgesetze} & $A_0 = E_0 \wedge 0 =$ & $A_0 = E_0 \vee 1 =$\\ \hline
\textbf{Dualitätsgesetze} & $A_0 = \neg 0 =$ & $A_0 = \neg 1 =$\\ \hline
\textbf{Neutralitätsgesetze} & $A_0 = E_0 \wedge 1 =$ & $A_0 = E_0 \vee 0 =$\\ \hline
\textbf{Idempotenzgesetze} & $A_0 = E_0 \wedge E_0 =$ & $A_0 = E_0 \vee E_0 =$\\ \hline
\textbf{Kommutativgesetze} & $A_0 = E_0 \wedge E_1 =$ & $A_0 = E_0 \vee E_1 =$\\ \hline
\end{tblr}
\caption{Gesetze der booleschen Algebra (Teil 1).}
\end{table}
\end{exercise}
\begin{solution}
\begin{table}[H]
\centering
\begin{tblr}{
colspec={|c|Q[c, l, 4cm]|Q[c, l, 4cm]|},
stretch=1.5
}
\hline
\textbf{Extremalgesetze} & $A_0 = E_0 \wedge 0 = 0$ & $A_0 = E_0 \vee 1 = 1$\\ \hline
\textbf{Dualitätsgesetze} & $A_0 = \neg 0 = 1$ & $A_0 = \neg 1 = 0$\\ \hline
\textbf{Neutralitätsgesetze} & $A_0 = E_0 \wedge 1 = E_0$ & $A_0 = E_0 \vee 0 = E_0$\\ \hline
\textbf{Idempotenzgesetze} & $A_0 = E_0 \wedge E_0 = E_0$ & $A_0 = E_0 \vee E_0 = E_0$\\ \hline
\textbf{Kommutativgesetze} & $A_0 = E_0 \wedge E_1 = E_1 \wedge E_0$ & $A_0 = E_0 \vee E_1 = E_1 \vee E_0$\\ \hline
\end{tblr}
\end{table}
\end{solution}
