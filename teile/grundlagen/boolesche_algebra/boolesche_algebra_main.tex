% !TEX root = ../../../main.tex

\toggletrue{image}
\togglefalse{imagehover}
\chapterimage{logic_boat_1}
\chapterimagetitle{\uppercase{Logic Boat 1}}
\chapterimageurl{https://xkcd.com/1134/}
\toggletrue{imagehover}

\chapter{Boolesche Algebra}
\label{chapter-boolesche-algebra}

Die Grundlage für die Logikgatter bildet die \textbf{Boolesche Algebra}\footnote{Benannt nach Georg Boole, 1815 - 1864}. Für jedes Logikgatter gibt es ein \say{Rechensymbol}. Das Verhalten eines Logikgatters stellen wir dann als \textbf{Boolesche Formeln} dar. Die Lernziele lauten:\\

\newcommand{\boolescheAlgebraLernziele}{
\protect\begin{minipage}{\textwidth}
\begin{todolist}
\item Sie nennen für die drei Logikgatter \texttt{UND}, \texttt{ODER} und \texttt{NICHT} jeweils die Boolesche Operation.
\item Sie notieren für die drei Logikgatter jeweils den Booleschen Operator.
\item Sie stellen das Verhalten für die drei Logikgatter jeweils als Boolesche Formel dar.
\end{todolist}
\end{minipage}
}

\lernziel{\autoref{chapter-boolesche-algebra}, \nameref{chapter-boolesche-algebra}}{\protect\boolescheAlgebraLernziele}

\boolescheAlgebraLernziele

\section{Konjunktion}

Die Boolesche Operation für das \texttt{UND}-Gatter ist die \textbf{Konjunktion} der Booleschen Algebra. Wir notieren die Boolesche Formel für das Verhalten des \texttt{UND}-Gatters wie folgt:

\begin{center}
\Large
$\underbrace{A_0 = \overbrace{E_0 \wedge E_1}^{\text{Konjunktion}}}_{\text{Boolesche Formel}}$
\end{center}

Das $\wedge$-Symbol ist der Boolesche Operator für die Konjunktion. Die Formel wird gesprochen als \say{$A_0$ gleich $E_0$ \textbf{und} $E_1$}.

\begin{hinweis}
Sie kennen den Begriff des Operators bereits aus der Mathematik. Bei der Addition ist $+$ ein (arithmetische) Operator. Meist einfach nur Rechensymbol genannt.
\end{hinweis}

\section{Disjunktion}

Die Boolesche Operation für das \texttt{ODER}-Gatter ist die \textbf{Disjunktion} der Booleschen Algebra. Wir notieren die Boolesche Formel für das Verhalten des \texttt{ODER}-Gatters wie folgt:

\begin{center}
\Large
$\underbrace{A_0 = \overbrace{E_0 \vee E_1}^{\text{Disjunktion}}}_{\text{Boolesche Formel}}$
\end{center}

Das $\vee$-Symbol ist der Boolesche Operator für die Disjunktion. Die Formel wird gesprochen als \say{$A_0$ gleich $E_0$ \textbf{oder} $E_1$}.

\section{Negation}

Die Boolesche Operation für das \texttt{NICHT}-Gatter ist die \textbf{Negation} der Booleschen Algebra. Wir notieren die Boolesche Formel für das Verhalten des \texttt{NICHT}-Gatters wie folgt:

\begin{center}
\Large
$\underbrace{A_0 = \overbrace{\neg E_0}^{\text{Negation}}}_{\text{Boolesche Formel}}$
\end{center}

Das $\neg$-Symbol ist der Boolesche Operator für die Negation. Die Formel wird gesprochen als \say{$A_0$ gleich \textbf{nicht} $E_0$}.

\section{Übungen}

\begin{exercise}
Füllen Sie die folgende Tabelle aus. Verwenden Sie $E_0$, $E_1$ und $A_0$.

\begin{table}[htb]
\centering
\begin{tblr}{
colspec={|Q[c, m]|Q[c, m, 3.5cm]|Q[c, m, 3.5cm]|Q[c, m, 3.5cm]|}
}
\hline
& \textbf{UND} & \textbf{ODER} & \textbf{NICHT} \\ \hline
\textbf{Boolescher}\\ {\textbf{Operator}} &     \vspace{0.5cm}           &               &                \\ \hline
\textbf{Logikgatter}         &        \vspace{2cm}        &               &                \\ \hline
\end{tblr}
\end{table}
\end{exercise}

\begin{exercise}
Wir können, wie in der Mathematik, das Ergebnis einer Booleschen Formel für konkrete Eingangswerte (eine sogenannte Belegung) ausrechnen. Falls $E_0 = 0$ und $E_1 = 1$ ist, dann erhalten wir für $A_0 = E_0 \wedge E_1$ das Ergebnis aus \autoref{boolean-algebra-eq-1}. Das Ergebnis ergibt sich aus der Definition der Konjunktion. 
\begin{align}
A_0 = E_0 \wedge E_1 = 0 \wedge 1 = 0\label{boolean-algebra-eq-1}
\end{align}
Füllen Sie mit dieser Information folgende Tabelle aus.

\begin{table}[htb]
\centering
\begin{tblr}{
colspec={|c|Q[c, l, 4cm]|Q[c, l, 4cm]|},
stretch=2
}
\hline
\textbf{Neutralitätsgesetze} & $E_0 \wedge 1 =$ & $E_0 \vee 0 =$\\ \hline
\textbf{Extremalgesetze} & $E_0 \wedge 0 =$ & $E_0 \vee 1 =$\\ \hline
\textbf{Dualitätsgesetze} & $\neg 0 =$ & $\neg 1 =$\\ \hline
\textbf{Idempotenzgesetze} & $E_0 \wedge E_0 =$ & $E_0 \vee E_0 =$\\ \hline
\textbf{Kommutativgesetze} & $E_0 \wedge E_1 =$ & $E_0 \vee E_1 =$\\ \hline
\end{tblr}
\caption{Gesetze der Booleschen Algebra (Teil 1).}
\end{table} 
\end{exercise}
