% !TEX root = ../../../main.tex

\toggletrue{image}
\togglefalse{imagehover}
\chapterimage{logic_boat_2}
\chapterimagetitle{\uppercase{Logic Boat 2}}
\chapterimageurl{https://xkcd.com/1134/}
\toggletrue{imagehover}


\chapter{Wahrheitstabellen}
\label{chapter-wahrheitstabellen}

Um das Verhalten einer booleschen Formel oder eines Logikgatters besser zu verstehen, stellen wir alle Kombinationen von Ein- und Ausgängen in einer Tabelle dar. Die Lernziele sind:\\
 
\newcommand{\wahrheitstabellenLernziele}{
\protect\begin{minipage}{\textwidth}
\begin{todolist}
\item Sie erklären, was wir unter einer Wahrheitstabelle verstehen.
\item Sie notieren für jedes der drei Logikgatter \texttt{UND}, \texttt{ODER} und \texttt{NICHT} die Wahrheitstabelle.
\end{todolist}
\end{minipage}
}

\lernziel{\autoref{chapter-wahrheitstabellen}, \nameref{chapter-wahrheitstabellen}}{\protect\wahrheitstabellenLernziele}

\wahrheitstabellenLernziele

\leavevmode
\newline

\begin{definition}[Wahrheitstabelle]
Eine Wahrheitstabelle (engl. truth table) besteht aus Spalten und Zeilen. Für jeden Eingang und Ausgang gibt es eine Spalte. Jede Zeile enthält eine Kombination der Eingangswerte und den zugehörigen Ausgangswert.
\end{definition}

Eine Wahrheitstabelle zeigt den Zusammenhang zwischen Ein- und Ausgängen. Die Wahrheitstabelle ist also eine andere Darstellung einer booleschen Formel bzw. eines Logikgatters. Im Folgenden werden wir die Wahrheitstabellen für die drei Grundgatter bzw. die zugehörigen booleschen Formeln dargestellt.

\section{Konjunktion / \texttt{UND}-Gatter}

Für die boolesche Formel  $A_0=E_0 \wedge E_1$ (bzw. das \texttt{UND}-Gatter), ergibt sich eine Wahrheitstabelle mit vier Zeilen (siehe \autoref{and-switcher-table-1}). Es gibt vier mögliche Kombinationen für die Einstellung der Schalter (Eingänge $E_0$ und $E_1$).

\begin{table}[ht]
\begin{minipage}{0.45\textwidth}
\centering
\begin{tblr}{|c|c||c|}
\hline
$E_1$ & $E_0$ & $A_0 = E_0 \wedge E_1$ \\ \hline[2pt]
Nicht gedrückt    &  Nicht gedrückt     & Aus    \\ \hline
Nicht gedrückt     & Gedrückt    & Aus   \\ \hline
Gedrückt   & Nicht gedrückt      & Aus   \\ \hline
Gedrückt    & Gedrückt    & An    \\ \hline
\end{tblr}
\caption{Vollständige Darstellung.} 
\label{and-switcher-table-1}
\end{minipage}
\hfill
\begin{minipage}{0.45\textwidth}
\centering
\begin{tblr}{|c|c||c|}
\hline
$E_1$ & $E_0$ & $A_0 = E_0 \wedge E_1$ \\ \hline[2pt]
Strom aus    &  Strom aus    & Strom aus    \\ \hline
Strom aus     & Strom ein    & Strom aus   \\ \hline
Strom ein   & Strom aus     & Strom aus   \\ \hline
Strom ein    & Strom ein    & Strom ein    \\ \hline
\end{tblr}
\caption{Einheitliche Repräsentation.} 
\label{and-switcher-table-2}
\end{minipage}
\end{table}

Wir können den Inhalt der \autoref{and-switcher-table-1} auch verallgemeinern (abstrahieren). Ein Schalter kontrolliert, ob Strom fliesst oder nicht. Eine Glühlampe zeigt an, ob Strom fliesst oder nicht. \autoref{and-switcher-table-2} zeigt die einheitliche Notation. Aber auch diese Notation ist sehr aufwendig. Wir können die Schreibarbeit abkürzen, indem wir für \textbf{\say{Strom aus} eine $0$} notieren und für \textbf{\say{Strom ein} eine $1$}. So können wir \autoref{and-switcher-table-1} noch einfacher darstellen. \autoref{and-ttt-1} ist die sogenannte \textbf{Wahrheitstabelle}, wie sie typischerweise in der Digitaltechnik notiert wird.

\begin{table}[htb]
\centering
\begin{tblr}{|c|c||c|}
\hline
$E_1$ 	& $E_0$	& $A_0 = E_0 \wedge E_1$ \\ \hline[2pt]
$0$		& $0$   & 0 \\ \hline
$0$		& $1$   & 0 \\ \hline	
$1$ 		& $0$   & 0 \\ \hline 	
$1$		& $1$   & 1 \\ \hline	
\end{tblr}
\caption{Die Wahrheitstabelle für die Konjunktion bzw. das \texttt{UND}-Gatter.} 
\label{and-ttt-1}
\end{table}

Eine Wahrheitstabelle zeigt alle möglichen Kombinationen für die Einstellung der Eingänge.

\begin{definition}[Belegung]
Die konkrete Zuordnung der Binärziffern ($0$ und $1$) zu den \textbf{Eingängen} wird als Eingangsbelegung oder kurz \textbf{Belegung} bezeichnet.
\end{definition}

Jede \textbf{Zeile einer Wahrheitstabelle} enthält also eine \textbf{Belegung der Eingänge}. Die Belegung der Eingänge bestimmt den Wert des Ausgangs.

\section{Disjunktion / \texttt{ODER}-Gatter}

Für die boolesche Formel  $A_0=E_0 \vee E_1$ (bzw. das \texttt{ODER}-Gatter) ergibt sich die Wahrheitstabelle aus \autoref{or-ttt}.

\begin{table}[htb]
\centering
\begin{tblr}{|c|c||c|}
\hline
$E_1$ 	& $E_0$	& $A_0 = E_0 \vee E_1$ \\ \hline[2pt]
$0$		& $0$   & 0 \\ \hline
$0$		& $1$   & 1 \\ \hline	
$1$ 		& $0$   & 1 \\ \hline 	
$1$		& $1$   & 1 \\ \hline	
\end{tblr}
\caption{Die Wahrheitstabelle für die Disjunktion bzw. das \texttt{ODER}-Gatter.} 
\label{or-ttt}
\end{table}

\section{Negation / \texttt{NICHT}-Gatter}

Für die Boolesche Formel  $A_0=\neg E_0$ (bzw. das \texttt{NICHT}-Gatter) ergibt sich eine Wahrheitstabelle mit zwei Zeilen. Für den Eingang $E_0$ gibt es zwei mögliche Belegungen. Es ergibt sich die Wahrheitstabelle aus \autoref{not-ttt}.

\begin{table}[htb]
\centering
\begin{tblr}{|c||c|}
\hline
$E_0$ 	& $A_0 = \neg E_0$ \\ \hline[2pt]
$0$		& 1 \\ \hline
$1$		& 0 \\ \hline	
\end{tblr}
\caption{Die Wahrheitstabelle für die Negation bzw. das \texttt{NICHT}-Gatter.} 
\label{not-ttt}
\end{table}

\newpage

\section{Übungen}

\begin{exercise}
Füllen Sie die nachstehende Tabelle aus. Verwenden Sie $E_0$, $E_1$ und $A_0$.

\begin{table}[htb]
\centering
\begin{tblr}{
colspec={|Q[c, m]|[2pt]Q[c, m, 3.5cm]|[2pt]Q[c, m, 3.5cm]|[2pt]Q[c, m, 3.5cm]|}
}
\hline
& \textbf{UND} & \textbf{ODER} & \textbf{NICHT} \\ \hline[2pt]
\textbf{\small Boolesche Formel} & $A_0 = E_0 \wedge E_1$ & $A_0 = E_0 \vee E_1$ & $A_0 = \neg E_0$            \\ \hline
\textbf{\small Logikgatter} &        
\begin{circuitikz}
\draw (0,0) node[and port] (AND1) {}
(AND1.in 1) node[anchor=east] {$E_0$} 
(AND1.in 2) node[anchor=east] {$E_1$}
(AND1.out) node[anchor=west] {$A_0$};
\end{circuitikz}
      &               
      \begin{circuitikz}
\draw (0,0) node[or port] (OR1) {}
(OR1.in 1) node[anchor=east] {$E_0$} 
(OR1.in 2) node[anchor=east] {$E_1$}
(OR1.out) node[anchor=west] {$A_0$};
\end{circuitikz}
      &
      \begin{circuitikz}
\draw (0,0) node[european not port] (NOT1) {}
(NOT1.in) node[anchor=east] {$E_0$} 
(NOT1.out) node[anchor=west] {$A_0$};
\end{circuitikz}
      \\ \hline
\textbf{\small Wahrheitstabelle}         &     \vspace{2.5cm}         &               &                \\ \hline
\end{tblr}
\end{table}
\end{exercise}

\vspace{-0.75cm}

\begin{exercise}
Das \texttt{UND}- und \texttt{ODER}-Gatter gibt es auch mit mehr als zwei Eingängen (Logicly: bis zu \num{8} Eingänge pro Gatter). Füllen Sie die Wahrheitstabelle  für die beiden Ausgänge aus.

\begin{table}[htb]
\centering
\begin{tblr}{|c|c|c||c|c|}
\hline
$E_2$ & $E_1$	& $E_0$ & $A_0 = E_0 \wedge E_1 \wedge E_2$ & $A_1 = E_0 \vee E_1 \vee E_2$ \\ \hline[2pt]
$0$ & $0$ & $0$ & & \\ \hline
$0$ & $0$ & $1$ & & \\ \hline
$0$ & $1$ & $0$ & & \\ \hline
$0$ & $1$ & $1$ & & \\ \hline
$1$ & $0$ & $0$ & & \\ \hline
$1$ & $0$ & $1$ & & \\ \hline
$1$ & $1$ & $0$ & & \\ \hline
$1$ & $1$ & $1$ & & \\ \hline
\end{tblr}
\end{table}
\end{exercise}

\vspace{-0.75cm}

\begin{exercise}
Untersuchen Sie für alle bisherigen Wahrheitstabellen, die \textbf{Anzahl der Eingänge} und die \textbf{Anzahl der verschiedenen Belegungen} (Anzahl der Zeilen). Tragen Sie dann das Ergebnis in die nachfolgenden Tabelle ein. Können Sie auch für vier, fünf und sechs Eingänge die Anzahl bestimmen? Wie lautet die Formel für $n$ Eingänge? 

\begin{table}[htb]
\centering
\begin{tblr}{|c|c|}
\hline
\textbf{Anzahl Eingänge} & \textbf{Anzahl Belegungen} \\ \hline[2pt]
1 &\\ \hline
2 &\\ \hline
3 &\\ \hline
4 &\\ \hline
5 &\\ \hline
6 &\\ \hline
\end{tblr}
\end{table}

Formel:
\fillwithgrid{\stretch{1}}
\end{exercise}

\newpage

\begin{exercise}
Studieren Sie abermals alle bisherigen Wahrheitstabellen. Legen Sie dieses Mal den Fokus auf die Belegungen. Versuchen Sie daraus abzuleiten, wie Sie \textbf{systematisch} alle möglichen Belegungen für die Eingänge auflisten können. Erstellen Sie dann die Wahrheitstabellen für vier und fünf Eingänge mit allen Belegungen.

\begin{table}[H]
\centering
\begin{minipage}{0.35\textwidth}
\centering
\begin{tblr}{|c|[4pt]c|c|[4pt]c|c|c|}
\hline
$E_0$ & $E_1$ & $E_0$ & $E_2$ & $E_1$	& $E_0$ \\ \hline[2pt]
$0$ & $0$ & $0$ & $0$ & $0$ & $0$ \\ \hline
$1$ & $0$ & $1$ & $0$ & $0$ & $1$ \\ \hline
& $1$ & $0$ & $0$ & $1$ & $0$ \\ \hline
& $1$ & $1$ & $0$ & $1$ & $1$ \\ \hline
& & & $1$ & $0$ & $0$ \\ \hline
& & & $1$ & $0$ & $1$ \\ \hline
& & & $1$ & $1$ & $0$ \\ \hline
& & & $1$ & $1$ & $1$ \\ \hline
\end{tblr}
\caption{Drei Wahrheitstabellen.}
\vspace{1cm}
\begin{tblr}{|Q[c, m, 1cm]|Q[c, m, 1cm]|Q[c, m, 1cm]|Q[c, m, 1cm]|}
\hline
$E_3$ & $E_2$	& $E_1$ & $E_0$ \\ \hline[2pt]
& & & \\ \hline
& & & \\ \hline
& & & \\ \hline
& & & \\ \hline
& & & \\ \hline
& & & \\ \hline
& & & \\ \hline
& & & \\ \hline
& & & \\ \hline
& & & \\ \hline
& & & \\ \hline
& & & \\ \hline
& & & \\ \hline
& & & \\ \hline
& & & \\ \hline
& & & \\ \hline
\end{tblr}
\end{minipage}
\hfill
\begin{minipage}{0.6\textwidth}
\centering
\begin{tblr}{|Q[c, m, 1cm]|Q[c, m, 1cm]|Q[c, m, 1cm]|Q[c, m, 1cm]|Q[c, m, 1cm]|}
\hline
$E_4$ & $E_3$	& $E_2$ & $E_1$ & $E_0$ \\ \hline[2pt]
& & & & \\ \hline
& & & & \\ \hline
& & & & \\ \hline
& & & & \\ \hline
& & & & \\ \hline
& & & & \\ \hline
& & & & \\ \hline
& & & & \\ \hline
& & & & \\ \hline
& & & & \\ \hline
& & & & \\ \hline
& & & & \\ \hline
& & & & \\ \hline
& & & & \\ \hline
& & & & \\ \hline
& & & & \\ \hline
& & & & \\ \hline
& & & & \\ \hline
& & & & \\ \hline
& & & & \\ \hline
& & & & \\ \hline
& & & & \\ \hline
& & & & \\ \hline
& & & & \\ \hline
& & & & \\ \hline
& & & & \\ \hline
& & & & \\ \hline
& & & & \\ \hline
& & & & \\ \hline
& & & & \\ \hline
& & & & \\ \hline
& & & & \\ \hline
\end{tblr}
\end{minipage}
\end{table}
\end{exercise}
