% !TEX root = ../../../main.tex

\toggletrue{image}
\toggletrue{imagehover}
\chapterimage{karnaugh}
\chapterimagetitle{\uppercase{Valentine - Karnaugh}}
\chapterimageurl{https://xkcd.com/62/}
\chapterimagehover{'Love and circuit analysis, hand in hand at last.'}

\chapter{\acs{KV}-Diagramme}
\label{chapter-kv-diagramme}

Boolesche Formeln in \ac{DNF} können sehr umfangreich sein. Wir zeigen nun, wie wir mit einem grafischen Verfahren eine vereinfachte Formel in \ac{DNF} erstellen können. Die Lernziele lauten:\\

\newcommand{\kvDiagrammeLernziele}{
\protect\begin{minipage}{\textwidth}
\begin{todolist}
\item Sie erstellen eine Boolesche Formeln in \ac{DNF} mithilfe eines \acs{KV}-Diagramms.
\end{todolist}
\end{minipage}
}

\lernziel{\autoref{chapter-kv-diagramme}, \nameref{chapter-kv-diagramme}}{\protect\kvDiagrammeLernziele}

\kvDiagrammeLernziele

\section{Was heisst vereinfachen?}

Wir möchten weiterhin Boolesche Formeln in \acs{DNF} aus einer Wahrheitstabelle erstellen. Jedoch soll die \textbf{Anzahl} der \textbf{Konjunktionen} und \textbf{Disjunktionen} in der Formel möglichst gering sein.
\begin{example}
Die Boolesche Formel $A_0 = (E_0 \wedge E_1 \wedge E_2) \vee (E_0 \wedge \neg E_1 \wedge E_2) \vee (E_0 \wedge E_1 \wedge \neg E_2)$ können wir zum Beispiel durch die Anwendung der Booleschen Gesetze vereinfachen. Wir erhalten:
\begin{center}
$A_0 = (E_0 \wedge E_1 \wedge E_2) \vee (E_0 \wedge \neg E_1 \wedge E_2) \vee (E_0 \wedge E_1 \wedge \neg E_2) = (E_0 \wedge E_2) \vee (E_0 \wedge E_1)$
\end{center}
\end{example}

\section{Konstruktion einer disjunktiven Normalform mit einem \acs{KV}-Diagramm}

Wir können für \textbf{jede} Wahrheitstabelle mit eine vereinfachte Boolesche Formel erstellen:

\begin{important}[Eine vereinfachte Boolesche Formel in \acs{DNF} erstellen:]
Für jeden Ausgang $A$:
\begin{enumerate}
\item[1.)] Bestimme die Anzahl der Eingänge ($=e$) in der Wahrheitstabelle.
\item[2.)] Bestimme die Anzahl der Zeilen ($=z$) in der Wahrheitstabelle.
\item[3.)] Verwende das \acs{KV}-Diagramm für $e$ Eingänge mit $z$ Zellen.
\item[4.)] Suche alle Zeilen in der Wahrheitstabelle, bei denen der Ausgang $1$ ist.
\item[5.)] Übertrage jede $1$ in das \acs{KV}-Diagramm in die dazugehörige Zelle.
\item[6.)] Bilde im \acs{KV}-Diagramm möglichst grosse Blöcke. Die Blockgrösse muss eine $2$er-Potenz sein (d.h. $1$, $2$, $4$, $8$, $\dots$).
\item[7.)] Erstelle pro Block eine Teilformel mithilfe von Negierungen und Konjunktionen.
\item[8.)] Für den Ausgang $A$: verknüpfe die Teilformeln aus 7.) mit Disjunktionen.
\end{enumerate}
\end{important}

Wir zeigen nun an einem Beispiel, wie wir mithilfe eines \ac{KV}-Diagramms für eine gegebene Wahrheitstabelle die dazugehörige Boolesche Formel in \ac{DNF} erzeugen kann.

\begin{example}
Die Wahrheitstabelle aus \autoref{table-kv-bsp-1} hat drei Eingänge und $2^3 = 8$ Zeilen. Wir verwenden deshalb das dazugehörige \ac{KV}-Diagramm für drei Eingänge, welches aus acht Zellen besteht.

\begin{figure}[htb]
\centering
\begin{minipage}{0.45\textwidth}
\centering
\begin{tblr}{|c|c|c||c|}
\hline
$E_2$ & $E_1$ & $E_0$ & $A_0 = \text{???}$ \\ \hline[2pt]
0 & 0 & 0 & 1\\ \hline
0 & 0 & 1 & 1\\ \hline
0 & 1 & 0 & 0\\ \hline
0 & 1 & 1 & 1\\ \hline
1 & 0 & 0 & 0\\ \hline
1 & 0 & 1 & 0\\ \hline
1 & 1 & 0 & 0\\ \hline
1 & 1 & 1 & 1\\ \hline
\end{tblr}
\caption{Wahrheitstabelle für $A_0$.}
\label{table-kv-bsp-1}
\end{minipage}
\hfill
\begin{minipage}{0.45\textwidth}
\centering
\begin{tikzpicture}[karnaugh]
  \karnaughmap{3}{$A_0$}{{$E_2$}{$E_1$}{$E_0$}}{}{
     \node (negE01) at (0.5, 2.25) {\tiny $\neg E_0$};
     \node (negE02) at (3.5, 2.25) {\tiny $\neg E_0$};
     \node (negE11) at (-0.35, 1.5) {\tiny $\neg E_1$};
     \node (negE21) at (0.5, -.25) {\tiny $\neg E_2$};
     \node (negE22) at (1.5, -.25) {\tiny $\neg E_2$};
  }
\end{tikzpicture}
\caption{\acs{KV}-Diagramm für drei Eingänge. Es ist auch üblich, die negierten Eingänge nicht explizit darzustellen.}
\label{figure-kv-diagramm-bsp-1}
\end{minipage}
\end{figure}

Nun suchen wir in der Wahrheitstabelle diejenigen Zeilen, welche am Ausgang eine $1$ besitzen. Wir übertragen dann diese Einsen in die passende Zelle im \ac{KV}-Diagramm.

\begin{figure}[htb]
\centering
\begin{minipage}{0.45\textwidth}
\centering
\begin{tblr}{
colspec = {|c|c|c||c|c},
cell{2}{4} = {yellow!25},
cell{3}{4} = {red!25},
cell{5}{4} = {blue!25},
cell{9}{4} = {green!25}
}
\cline{1-4}
$E_2$ & $E_1$ & $E_0$ & $A_0 = \text{???}$ \\ \cline[2pt]{1-4}
0 & 0 & 0 & 1 &	\circled{1} \\ \cline{1-4}
0 & 0 & 1 & 1 &	\circled{2} \\ \cline{1-4}
0 & 1 & 0 & 0 &	\\ \cline{1-4}
0 & 1 & 1 & 1 &	\circled{3} \\ \cline{1-4}
1 & 0 & 0 & 0 &	\\ \cline{1-4}
1 & 0 & 1 & 0 &	\\ \cline{1-4}
1 & 1 & 0 & 0 &	\\ \cline{1-4}
1 & 1 & 1 & 1 &	\circled{4} \\ \cline{1-4}
\end{tblr}
\caption{Die Eingänge bestimmen die Position im \acs{KV}-Diagramm.}
\label{table-kv-bsp-2}
\end{minipage}
\hfill
\begin{minipage}{0.45\textwidth}
\centering
\begin{tikzpicture}[karnaugh, grp/.style n args={3}{#1, fill=#1!50,
                      minimum width=#2\kmunitlength,
                      minimum height=#3\kmunitlength,
                      rounded corners=0.2\kmunitlength,
                      fill opacity=0.6,
                      rectangle, draw}]
  \karnaughmap{3}{$A_0$}{{$E_2$}{$E_1$}{$E_0$}}{11~1 ~~~1}{
     \node (negE01) at (0.5, 2.25) {\tiny $\neg E_0$};
     \node (negE02) at (3.5, 2.25) {\tiny $\neg E_0$};
     \node (negE11) at (-0.35, 1.5) {\tiny $\neg E_1$};
     \node (negE21) at (0.5, -.25) {\tiny $\neg E_2$};
     \node (negE22) at (1.5, -.25) {\tiny $\neg E_2$};
   \node[grp={yellow}{0.9}{0.9}](c000) at (0.5, 1.5) {};
   \node[grp={red}{0.9}{0.9}](c001) at (1.5, 1.5) {};
   \node[grp={blue}{0.9}{0.9}](c011) at (1.5, 0.5) {};
   \node[grp={green}{0.9}{0.9}](c011) at (2.5, 0.5) {};
  }
\end{tikzpicture}
\caption{Jede farbige Zelle entspricht einer Zeile in der Wahrheitstabelle. Wir können die Position der Einsen dadurch herausfinden, in dem wir die Teilformel der entsprechenden Zeile betrachten, so wie wir es bei der \protect\say{normalen} Konstruktion einer Booleschen Formeln in \protect\acs{DNF} kennengelernt haben.}
\label{figure-kv-diagramm-bsp-2}
\end{minipage}
\end{figure}

Die Zeile \circled{3} in der Wahrheitstabelle entspricht zum Beispiel der Teilformel $\neg E_2 \wedge E_1 \wedge E_0$. Wir müssen also im \acs{KV}-Diagramm die Zelle suchen, die durch $\neg E_2$, $E_1$ und $E_0$ abgedeckt ist. Dies ist durch die blaue Zelle in \autoref{figure-kv-diagramm-bsp-2} gegeben. Für die Zeilen \circled{1}, \circled{2} und \circled{4} können wir auf die gleiche Weise verfahren.

\begin{important}
Wir übertragen \textbf{keine} Nullen in das \ac{KV}-Diagramm!
\end{important}

Nachdem alle Einsen korrekt übertragen sind, müssen wir möglichst \textbf{grosse Blöcke} bilden. Ein Block ist eine Zusammenfassung von Einsen. Dabei muss die Anzahl der Einsen pro Block eine $2$er-Potenz sein ($1$, $2$, $4$, $8$, $\dots$). Im \ac{KV}-Diagramm aus \autoref{figure-kv-diagramm-bsp-2} schaffen wir es nicht, alle Einsen in einen Block zu packen. Wir bilden deshalb zwei $2$er-Blöcke (siehe \autoref{figure-kv-diagramm-bsp-3}).

\begin{figure}[htb]
\centering
\begin{tikzpicture}[karnaugh, grp/.style n args={3}{#1,
                      minimum width=#2\kmunitlength,
                      minimum height=#3\kmunitlength,
                      rounded corners=0.2\kmunitlength,
                      rectangle, draw, very thick}]
  \karnaughmap{3}{$A_0$}{{$E_2$}{$E_1$}{$E_0$}}{11~1 ~~~1}{
  	\node[grp={red}{1.8}{0.9}](g1) at (1, 1.5) {};
	\node[grp={red}{1.8}{0.9}](g2) at (2, 0.5) {};
	\node (negE01) at (0.5, 2.25) {\tiny $\neg E_0$};
	\node (negE02) at (3.5, 2.25) {\tiny $\neg E_0$};
	\node (negE11) at (-0.35, 1.5) {\tiny $\neg E_1$};
	\node (negE21) at (0.5, -.25) {\tiny $\neg E_2$};
     	\node (negE22) at (1.5, -.25) {\tiny $\neg E_2$};
  }
  \draw[latex-, thick] (1.75,1.5) -- (5,1.5)  node[anchor=west] {$T_1$};
  \draw[latex-, thick] (2.75,0.5) -- (5,0.5)  node[anchor=west] {$T_2$};
\end{tikzpicture}
\caption{Die beiden $2$er-Blöcke umfassen jeweils zwei Zellen.}
\label{figure-kv-diagramm-bsp-3}
\end{figure}

\newpage

Für jeden Block bilden wir nun \textbf{eine Teilformel}. Pro Block bestimmen wir alle \textbf{gemeinsamen Eingänge} und verknüpfen diese mit einer \textbf{Konjunktion}. Anschliessend werden die Teilformeln mit einer \textbf{Disjunktion} zur Booleschen Formel in \ac{DNF} für $A_0$ zusammengesetzt.

\begin{itemize}
\item $T_1 = \neg E_1 \wedge \neg E_2$
\begin{itemize}
\item $\neg E_0 \notin T_1$: nur die \textbf{erste} Eins in diesem Block \say{besitzt} Eingang $\neg E_0$
\item $E_0 \notin T_1$: nur die \textbf{zweite} Eins in diesem Block \say{besitzt} Eingang $E_0$
\item $E_1 \notin T_1$ und $E_2 \notin T_1$: keine Eins \say{besitzt} in diesem Block einen dieser Eingänge
\end{itemize}
\item $T_2 = E_0 \wedge E_1$
\end{itemize}

Daraus ergibt sich dann folgendes Ergebnis:
\begin{align*}
A_0 = (\neg E_1 \wedge \neg E_2) \vee (E_0 \wedge E_1)
\end{align*}

\end{example}

\section{\acs{KV}-Diagramme als Vorlage}

Wir beschränken uns hier auf die Verwendung von \ac{KV}-Diagrammen mit bis zu \num{4} Eingängen.

\begin{figure}[htb]
\centering
\begin{minipage}{0.45\textwidth}
\centering
\begin{tikzpicture}[karnaugh]
\karnaughmap{1}{$A$}{{$E_0$}}{}{
	\node (negE0) at (0.5, 1.25) {\tiny $\neg E_0$};
  }
\end{tikzpicture}
\caption{\num{1} Eingang}
\label{figure-kv-diagramm-e-1}
\end{minipage}
\hfill
\begin{minipage}{0.45\textwidth}
\centering
\begin{tikzpicture}[karnaugh]
\karnaughmap{2}{$A$}{{$E_0$}{$E_1$}}{}{
	\node (negE1) at (0.5, 2.25) {\tiny $\neg E_1$};
	\node (negE0) at (-0.35, 1.5) {\tiny $\neg E_0$};
  }
\end{tikzpicture}
\caption{\num{2} Eingänge}
\label{figure-kv-diagramm-e-2}
\end{minipage}
\end{figure}

\begin{figure}[htb]
\begin{minipage}{0.45\textwidth}
\centering
\begin{tikzpicture}[karnaugh]
\karnaughmap{3}{$A$}{{$E_0$}{$E_1$}{$E_2$}}{}{
	\node (negE01) at (0.5, 2.25) {\tiny $\neg E_0$};
	\node (negE02) at (3.5, 2.25) {\tiny $\neg E_0$};
	\node (negE11) at (-0.35, 1.5) {\tiny $\neg E_1$};
	\node (negE21) at (0.5, -.25) {\tiny $\neg E_2$};
     	\node (negE22) at (1.5, -.25) {\tiny $\neg E_2$};
  }
\end{tikzpicture}
\caption{\num{3} Eingänge}
\label{figure-kv-diagramm-e-3}
\end{minipage}
\hfill
\begin{minipage}{0.45\textwidth}
\centering
\begin{tikzpicture}[karnaugh]
\karnaughmap{4}{$A$}{{$E_0$}{$E_1$}{$E_2$}{$E_3$}}{}{
	\node (negE01) at (4.35, 3.5) {\tiny $\neg E_0$};
	\node (negE02) at (4.35, 2.5) {\tiny $\neg E_0$};
	\node (negE11) at (0.5, -.25) {\tiny $\neg E_1$};
     	\node (negE12) at (1.5, -.25) {\tiny $\neg E_1$};
	\node (negE12) at (-0.35, 3.5) {\tiny $\neg E_2$};
	\node (negE22) at (-0.35, 0.5) {\tiny $\neg E_2$};
	\node (negE31) at (0.5, 4.25) {\tiny $\neg E_3$};
	\node (negE32) at (3.5, 4.25) {\tiny $\neg E_3$};
  }
\end{tikzpicture}
\caption{\num{4} Eingänge}
\label{figure-kv-diagramm-e-4}
\end{minipage}
\end{figure}

\newpage

\section{Regeln zur Blockbildung}

Wir fassen hier die wichtigsten Vorschriften zur Blockbildung nochmals zusammen und zeigen ein paar spezielle Möglichkeiten für eine Blockbildung.

\begin{enumerate}
\item Blöcke dürfen \textbf{nicht} diagonal gebildet werden.
\item Die Blockgrösse muss eine $2$er-Potenz sein.
\item Ein Block sollte so gross wie möglich sein.
\item Jede Eins muss \textbf{mindestens} in einem Block vorhanden sein.
\item Blöcke \textbf{dürfen} sich überlappen.
\item Blöcke \textbf{dürfen} (unter gewissen Bedingungen) über den \say{Rand} gebildet werden.
\item Je weniger Blöcke, desto besser.
\end{enumerate}

\begin{figure}[htb]
\centering
\begin{minipage}{0.45\textwidth}
\centering
\begin{tikzpicture}[karnaugh]
\karnaughmap{3}{$A$}{{$E_0$}{$E_1$}{$E_2$}}{11~~ 1111}{
	\node (negE01) at (0.5, 2.25) {\tiny $\neg E_0$};
	\node (negE02) at (3.5, 2.25) {\tiny $\neg E_0$};
	\node (negE11) at (-0.35, 1.5) {\tiny $\neg E_1$};
	\node (negE21) at (0.5, -.25) {\tiny $\neg E_2$};
     	\node (negE22) at (1.5, -.25) {\tiny $\neg E_2$};
  }
\end{tikzpicture}
\caption{\num{3} Eingänge}
\label{figure-kv-diagramm-e-3-block-1}
\end{minipage}
\hfill
\begin{minipage}{0.45\textwidth}
\centering
\begin{tikzpicture}[karnaugh]
\karnaughmap{3}{$A$}{{$E_0$}{$E_1$}{$E_2$}}{1~1~ 1~1~}{
	\node (negE01) at (0.5, 2.25) {\tiny $\neg E_0$};
	\node (negE02) at (3.5, 2.25) {\tiny $\neg E_0$};
	\node (negE11) at (-0.35, 1.5) {\tiny $\neg E_1$};
	\node (negE21) at (0.5, -.25) {\tiny $\neg E_2$};
     	\node (negE22) at (1.5, -.25) {\tiny $\neg E_2$};
  }
\end{tikzpicture}
\caption{\num{3} Eingänge}
\label{figure-kv-diagramm-e-3-block-2}
\end{minipage}
\end{figure}

\begin{figure}[htb]
\centering
\begin{minipage}{0.45\textwidth}
\centering
\begin{tikzpicture}[karnaugh]
\karnaughmap{4}{$A$}{{$E_0$}{$E_1$}{$E_2$}{$E_3$}}{0110 0110 0110 0110}{
	\node (negE01) at (4.35, 3.5) {\tiny $\neg E_0$};
	\node (negE02) at (4.35, 2.5) {\tiny $\neg E_0$};
	\node (negE11) at (0.5, -.25) {\tiny $\neg E_1$};
     	\node (negE12) at (1.5, -.25) {\tiny $\neg E_1$};
	\node (negE12) at (-0.35, 3.5) {\tiny $\neg E_2$};
	\node (negE22) at (-0.35, 0.5) {\tiny $\neg E_2$};
	\node (negE31) at (0.5, 4.25) {\tiny $\neg E_3$};
	\node (negE32) at (3.5, 4.25) {\tiny $\neg E_3$};
  }
\end{tikzpicture}
\caption{\num{4} Eingänge}
\label{figure-kv-diagramm-e-4-block-1}
\end{minipage}
\hfill
\begin{minipage}{0.45\textwidth}
\centering
\begin{tikzpicture}[karnaugh]
\karnaughmap{4}{$A$}{{$E_0$}{$E_1$}{$E_2$}{$E_3$}}{1000 1000 1000 1000}{
	\node (negE01) at (4.35, 3.5) {\tiny $\neg E_0$};
	\node (negE02) at (4.35, 2.5) {\tiny $\neg E_0$};
	\node (negE11) at (0.5, -.25) {\tiny $\neg E_1$};
     	\node (negE12) at (1.5, -.25) {\tiny $\neg E_1$};
	\node (negE12) at (-0.35, 3.5) {\tiny $\neg E_2$};
	\node (negE22) at (-0.35, 0.5) {\tiny $\neg E_2$};
	\node (negE31) at (0.5, 4.25) {\tiny $\neg E_3$};
	\node (negE32) at (3.5, 4.25) {\tiny $\neg E_3$};
  }
\end{tikzpicture}
\caption{\num{4} Eingänge}
\label{figure-kv-diagramm-e-4-block-2}
\end{minipage}
\end{figure}

\begin{figure}[htb]
\centering
\begin{minipage}{0.45\textwidth}
\centering
\begin{tikzpicture}[karnaugh]
\karnaughmap{4}{$A$}{{$E_0$}{$E_1$}{$E_2$}{$E_3$}}{1000 1000 0000 1000}{
	\node (negE01) at (4.35, 3.5) {\tiny $\neg E_0$};
	\node (negE02) at (4.35, 2.5) {\tiny $\neg E_0$};
	\node (negE11) at (0.5, -.25) {\tiny $\neg E_1$};
     	\node (negE12) at (1.5, -.25) {\tiny $\neg E_1$};
	\node (negE12) at (-0.35, 3.5) {\tiny $\neg E_2$};
	\node (negE22) at (-0.35, 0.5) {\tiny $\neg E_2$};
	\node (negE31) at (0.5, 4.25) {\tiny $\neg E_3$};
	\node (negE32) at (3.5, 4.25) {\tiny $\neg E_3$};
  }
\end{tikzpicture}
\caption{\num{4} Eingänge}
\label{figure-kv-diagramm-e-4-block-3}
\end{minipage}
\hfill
\begin{minipage}{0.45\textwidth}
\centering
\begin{tikzpicture}[karnaugh]
\karnaughmap{4}{$A$}{{$E_0$}{$E_1$}{$E_2$}{$E_3$}}{1011 1010 1011 1010}{
	\node (negE01) at (4.35, 3.5) {\tiny $\neg E_0$};
	\node (negE02) at (4.35, 2.5) {\tiny $\neg E_0$};
	\node (negE11) at (0.5, -.25) {\tiny $\neg E_1$};
     	\node (negE12) at (1.5, -.25) {\tiny $\neg E_1$};
	\node (negE12) at (-0.35, 3.5) {\tiny $\neg E_2$};
	\node (negE22) at (-0.35, 0.5) {\tiny $\neg E_2$};
	\node (negE31) at (0.5, 4.25) {\tiny $\neg E_3$};
	\node (negE32) at (3.5, 4.25) {\tiny $\neg E_3$};
  }
\end{tikzpicture}
\caption{\num{4} Eingänge}
\label{figure-kv-diagramm-e-4-block-4}
\end{minipage}
\end{figure}

\newpage

\section{Übungen}

Stellen Sie die Blöcke im \ac{KV}-Diagramm \textbf{sauber} dar und notieren Sie \textbf{pro Block} die \textbf{Teilformel}.

\begin{exercise}
Erstellen Sie für den Ausgang $A_0$ (siehe \autoref{table-exercise-kv-1}) mithilfe des passenden \ac{KV}-Diagramms eine Boolesche Formel in \ac{DNF}.

\begin{table}[htb]
\centering
\begin{minipage}{0.3\textwidth}
\centering
\begin{tblr}{|c|c|c||c|}
\hline
$E_2$ & $E_1$ & $E_0$ & $A_0$ \\ \hline[2pt]
0 & 0 & 0 & 1 \\ \hline
0 & 0 & 1 & 0 \\ \hline
0 & 1 & 0 & 1 \\ \hline
0 & 1 & 1 & 1 \\ \hline
1 & 0 & 0 & 0 \\ \hline
1 & 0 & 1 & 1 \\ \hline
1 & 1 & 0 & 1 \\ \hline
1 & 1 & 1 & 1 \\ \hline
\end{tblr}
\caption{Wahrheitstabelle}
\label{table-exercise-kv-1}
\end{minipage}
\hfill
\begin{minipage}{0.65\textwidth}
\centering
\fillwithgrid	{2.5in}
\end{minipage}
\end{table}
\fillwithgrid	{0.75in}
\end{exercise}

\begin{exercise}
Erstellen Sie für den Ausgang $A_0$ (siehe \autoref{table-exercise-kv-2}) mithilfe des passenden \ac{KV}-Diagramms eine Boolesche Formel in \ac{DNF}.
\begin{table}[htb]
\centering
\begin{minipage}{0.3\textwidth}
\centering
\begin{tblr}{|c|c|c||c|}
\hline
$E_2$ & $E_1$ & $E_0$ & $A_0$ \\ \hline[2pt]
0 & 0 & 0 & 0  \\ \hline
0 & 0 & 1 & 1 \\ \hline
0 & 1 & 0 & 1 \\ \hline
0 & 1 & 1 & 1 \\ \hline
1 & 0 & 0 & 0 \\ \hline
1 & 0 & 1 & 0  \\ \hline
1 & 1 & 0 & 1 \\ \hline
1 & 1 & 1 & 1 \\ \hline
\end{tblr}
\caption{Wahrheitstabelle}
\label{table-exercise-kv-2}
\end{minipage}
\hfill
\begin{minipage}{0.65\textwidth}
\centering
\fillwithgrid	{2.5in}
\end{minipage}
\end{table}
\fillwithgrid	{0.75in}
\end{exercise}

\begin{exercise}
Erstellen Sie für den Ausgang $A_0$ und $A_1$ (siehe \autoref{table-exercise-kv-3}) mithilfe der passenden \ac{KV}-Diagramme eine Boolesche Formel in \ac{DNF}.
\begin{table}[htb]
\centering
\begin{minipage}{0.3\textwidth}
\centering
\begin{tblr}{|c|c|c|c||c|c|}
\hline
$E_3$ & $E_2$ & $E_1$ & $E_0$ & $A_0$ & $A_1$ \\ \hline[2pt]
0 & 0 & 0 & 0 & 0 & 0 \\ \hline
0 & 0 & 0 & 1 & 0 & 0 \\ \hline
0 & 0 & 1 & 0 & 0 & 0 \\ \hline
0 & 0 & 1 & 1 & 0 & 0 \\ \hline
0 & 1 & 0 & 0 & 0 & 0  \\ \hline
0 & 1 & 0 & 1 & 0 & 0 \\ \hline
0 & 1 & 1 & 0 & 1 & 0 \\ \hline
0 & 1 & 1 & 1 & 1 & 0 \\ \hline
1 & 0 & 0 & 0 & 0 & 1 \\ \hline
1 & 0 & 0 & 1 & 0 & 1 \\ \hline
1 & 0 & 1 & 0 & 0 & 1 \\ \hline
1 & 0 & 1 & 1 & 0 & 1 \\ \hline
1 & 1 & 0 & 0 & 0 & 0 \\ \hline
1 & 1 & 0 & 1 & 0 & 0 \\ \hline
1 & 1 & 1 & 0 & 1 & 1 \\ \hline
1 & 1 & 1 & 1 & 1 & 1 \\ \hline
\end{tblr}
\caption{Wahrheitstabelle}
\label{table-exercise-kv-3}
\end{minipage}
\hfill
\begin{minipage}{0.65\textwidth}
\centering
\fillwithgrid	{5in}
\end{minipage}
\end{table}
\fillwithgrid	{\stretch{1}}
\end{exercise}
