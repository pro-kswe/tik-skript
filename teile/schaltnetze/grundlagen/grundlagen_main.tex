% !TEX root = ../../../main.tex

\toggletrue{image}
\toggletrue{imagehover}
\chapterimage{logic_boat_3}
\chapterimagetitle{\uppercase{Logic Boat 3}}
\chapterimageurl{https://xkcd.com/1134/}
\chapterimagehover{Or a cabbage, for that matter. The goat makes sense. Goats are fine.}

\chapter{Grundlagen}
\label{chapter-schaltnetze-grundlagen}

Damit wir komplexere elektronische Bauteile erstellen können, müssen wir Logikgatter miteinander kombinieren. Wir sprechen von \textbf{Schaltnetzen}. Die Lernziele lauten:\\

\newcommand{\schaltnetzeGrundlagenLernziele}{
\protect\begin{minipage}{\textwidth}
\begin{todolist}
\item Sie erklären, was ein Schaltnetz ist.
\item Sie nennen die Rangfolge der Booleschen Operatoren $\wedge$, $\vee$, und $\neg$.
\item Sie erstellen für eine Boolesche Formel das dazugehörige Schaltnetz.
\item Sie erstellen für eine Boolesche Formel die dazugehörige Wahrheitstabelle. 
\end{todolist}
\end{minipage}
}

\lernziel{\autoref{chapter-schaltnetze-grundlagen}, \nameref{chapter-schaltnetze-grundlagen}}{\protect\schaltnetzeGrundlagenLernziele}

\schaltnetzeGrundlagenLernziele

\section{Rangfolge der Booleschen Operatoren}

Aus der Mathematik kennen Sie bereits die Rangfolge der arithmetischen Operatoren ($+$, $\cdot$, $-$, $\div$) und Klammern. Durch diese Rangfolge ist eindeutig definiert, wie eine Berechnung durchzuführen ist. \autoref{eq:mathe:bsp} zeigt ein Beispiel.
\begin{align}
(2 + 4 \cdot 5) \div 3 \label{eq:mathe:bsp}
\end{align}
Durch die Rangfolge aus \autoref{arithmetische-operatoren-rangfolge} wissen wir, dass die Gleichung wie folgt zu berechnen ist:

\begin{enumerate}
\item Berechne das Produkt von $4 \cdot 5$ (\say{Punkt-vor-Strich-Regel}).
\item Addiere zum Produkt die Zahl $5$.
\item Dividiere nun die Summe durch $3$.
\end{enumerate}

Die Reihenfolge der Berechnungen können wir kompakter als $(2 + (4 \cdot 5)) \div 3$ notieren. Auch für \textbf{Boolesche Operatoren} gibt es eine \textbf{Rangfolge} (siehe \autoref{boolesche-operatoren-rangfolge}), damit bei \textbf{Booleschen Formeln} mit \textbf{mehr als einem Operator} die Reihenfolge eindeutig definiert ist.

\begin{figure}[ht]
\centering
\begin{minipage}{0.45\textwidth}
\centering
\begin{tblr}{|c|[2pt]c|}
\hline
\textbf{Operator} & \textbf{Rang} \\ \hline[2pt]
Klammern & $1$ \\ \hline
$\cdot$ und $\div$ & $2$ \\ \hline
$+$ und $-$  &  $3$ \\ \hline
\end{tblr}
\caption{Arithmetische Operatoren}
\label{arithmetische-operatoren-rangfolge}
\end{minipage}
\hfill
\begin{minipage}{0.45\textwidth}
\centering
\begin{tblr}{|c|[2pt]c|}
\hline
\textbf{Operator} & \textbf{Rang} \\ \hline[2pt]
Klammern & $1$ \\ \hline
$\neg$    &  $2$ \\ \hline
$\wedge$     & $3$ \\ \hline
$\vee$ & $4$ \\ \hline
\end{tblr}
\caption{Boolesche Operatoren}
\label{boolesche-operatoren-rangfolge}
\end{minipage}
\end{figure}

Wir zeigen im nächsten Abschnitt, wie wir mit der Rangfolge der Booleschen Operatoren umgehen.

\section{Konstruktion von Schaltnetzen}

Wir zeigen nun an einem Beispiel, wie wir ein Schaltnetz für eine Boolesche Formel erstellen.

\begin{example}
Wir übersetzen die \autoref{eq:bsp1} in ein Schaltnetz.
\begin{align}
A_0 = \neg E_0 \wedge E_1 \label{eq:bsp1}
\end{align}
Wir können für eine Formel zunächst die Klammern gemäss der Rangfolge der Booleschen Operatoren hinzufügen. Aus \autoref{eq:bsp1} entsteht \autoref{eq:bsp1-klammern}.
\begin{align}
A_0 = (\neg E_0) \wedge E_1 \label{eq:bsp1-klammern}
\end{align}
Für \autoref{eq:bsp1-klammern} erstellen wir das Schaltnetz.

\paragraph{Schaltnetz}

\autoref{eq:bsp1-klammern} besteht aus zwei Eingängen ($E_0$ und $E_1$). Wir zeichnen zunächst beide Eingänge.

\begin{important}
In einer Formel können mehr als 2 Eingänge vorkommen. Es kann auch sein, dass ein Eingang mehrfach verwendet wird. Wir zeichnen jeden Eingang jedoch nur \textbf{einmal}.
\end{important}

Wir zeichnen zuerst $\neg E_0$, da dieser Teil in einer Klammer steht und somit einen höheren Rang als $\wedge$ hat. Dies bedeutet, wir müssen von $E_0$ \textbf{ausgehend} ein \texttt{NICHT}-Gatter zeichnen. Wir erhalten folgendes Ergebnis:

\begin{figure}[htb]
\centering
\begin{circuitikz}
\draw (0,0) node[european not port] (NOT1) {}
(NOT1.in 1) node[anchor=east] {$E_0$}; 
\draw (NOT1.in 1) ++ (0, -1) node[anchor=east] {$E_1$};
\end{circuitikz}
\end{figure}

Wir benötigen ein \texttt{UND}-Gatter, da in der Formel $\wedge$ vorhanden ist. Links von $\wedge$ steht $(\neg E_0)$. Wir müssen somit den \textbf{Ausgang} von $(\neg E_0)$ mit einem \textbf{Eingang} des \texttt{UND}-Gatters verbinden. Auf der rechten Seite von $\wedge$ steht $E_1$. Wir können somit direkt von $E_1$ \textbf{ausgehend} die Leitung mit dem \textbf{anderen} Eingang des \texttt{UND}-Gatters verbinden. Wir erhalten das Schaltnetz aus \autoref{bsp1-schaltnetz-zwischenstand}.

\begin{figure}[htb]
\centering
\begin{minipage}{0.45\textwidth}
\centering
\begin{circuitikz}
\draw (0,0) node[not port] (NOT1) {}
(NOT1.in 1) node[anchor=east] {$E_0$}; 
\draw (NOT1.out) -- ++(0.5,0) node[and port, anchor=in 1] (AND1) {};
\draw (NOT1.in 1) ++ (0, -1) node[anchor=east] {$E_1$} -- (0.5,-1) |- (AND1.in 2);
\end{circuitikz}
\caption{$(\neg E_0) \wedge E_1$}
\label{bsp1-schaltnetz-zwischenstand}
\end{minipage}
\hfill
\begin{minipage}{0.5\textwidth}
\centering
\begin{circuitikz}
\draw (0,0) node[not port] (NOT1) {}
(NOT1.in 1) node[anchor=east] {$E_0$}; 
\draw (NOT1.out) -- ++(0.5,0) node[and port, anchor=in 1] (AND1) {};
\draw (NOT1.in 1) ++ (0, -1) node[anchor=east] {$E_1$} -- (0.5,-1) |- (AND1.in 2);
\draw (AND1.out) node[anchor=west] {$A_0$};
\end{circuitikz}
\caption{Schaltnetz für $A_0 = (\neg E_0) \wedge E_1$.}
\label{bsp1-schaltnetz}
\end{minipage}
\end{figure}

Abschliessend verbinden wir noch den \textbf{Ausgang} des \texttt{UND}-Gattes mit der Beschriftung $A_0$. Das finale Schaltnetz ist in \autoref{bsp1-schaltnetz} dargestellt.
\end{example}

\begin{important}
Falls eine Boolesche Formel bereits Klammern besitzt, dann werden diese nicht aufgelöst. Klammern haben immer den \textbf{höchsten} Rang.
\end{important}

\newpage

\subsection{Übungen}

Sie müssen für die Schaltnetze kein Lineal verwenden. Eine saubere Darstellung ist ausreichend.

\begin{exercise}
Übersetzen Sie die Boolesche Formel $A_0 = E_0 \vee \neg E_1$ in ein Schaltnetz.
\fillwithgrid{1in}
\end{exercise}

\begin{solution}
$A_0 = E_0 \vee \neg E_1 = E_0 \vee (\neg E_1)$
\begin{figure}[htb]
\centering
\begin{circuitikz}
\draw (0,0) node[buffer port] (A) {}
(A.in 1) node[anchor=east] {$E_1$}; 
\node at (A.bout) [ocirc, right]{};
\draw (A.out) -- ++(0.5,0) node[or port, anchor=in 2] (B) {};
\draw (A.in 1) ++ (0, 1) node[anchor=east] {$E_0$} -- (0.5,1) |- (B.in 1);
\draw (B.out) node[anchor=west] {$A_0$};
\end{circuitikz}
\end{figure}
\end{solution}

\begin{exercise}
\label{ex-nand-schaltnetz}
Übersetzen Sie die Boolesche Formel $A_0 = \neg (E_0 \wedge E_1)$ in ein Schaltnetz.
\fillwithgrid{1in}
\end{exercise}

\begin{solution}
\begin{figure}[htb]
\centering
\begin{circuitikz}
\draw node[and port] (A) {} 
(A.in 1) node[anchor=east] {$E_0$}
(A.in 2) node[anchor=east] {$E_1$}
(A.out) node[buffer port, anchor=west] (B) {}
(B.in 1) node[anchor=east] {}; 
\node at (B.bout) [ocirc, right]{};
\draw (B.out) node[anchor=west] {$A_0$};
\end{circuitikz}
\end{figure}
\end{solution}

\begin{exercise}
\label{ex-nor-schaltnetz}
Übersetzen Sie die Boolesche Formel $A_0 = \neg (E_0 \vee E_1)$ in ein Schaltnetz.
\fillwithgrid{1in}
\end{exercise}

\begin{solution}
\begin{figure}[htb]
\centering
\begin{circuitikz}
\draw node[or port] (A) {} 
(A.in 1) node[anchor=east] {$E_0$}
(A.in 2) node[anchor=east] {$E_1$}
(A.out) node[buffer port, anchor=west] (B) {}
(B.in 1) node[anchor=east] {}; 
\node at (B.bout) [ocirc, right]{};
\draw (B.out) node[anchor=west] {$A_0$};
\end{circuitikz}
\end{figure}
\end{solution}

\begin{exercise}
\label{ex-xor-schaltnetz}
Übersetzen Sie die Boolesche Formel $A_0 = E_0 \wedge \neg E_1 \vee \neg E_0 \wedge E_1$ in ein Schaltnetz.
\fillwithgrid{\stretch{1}}
\end{exercise}

\begin{solution}
$A_0 = E_0 \wedge \neg E_1 \vee \neg E_0 \wedge E_1 = (E_0 \wedge (\neg E_1)) \vee ((\neg E_0) \wedge E_1)$
\begin{figure}[htb]
\centering
\begin{circuitikz}
\draw (-2, 1) node[and port] (A) {};
\draw (-2, -1) node[and port] (B) {}; 
\draw (0,0) node[or port] (C) {}; 
\draw (A.out) |- (C.in 1);
\draw (B.out) |- (C.in 2);
\draw (C.out) node[anchor=west] {$A_0$};
\draw (-4,-0.5) node[buffer port] (D) {};
\node at (D.bout) [ocirc, right]{};
\draw (-4,0.5) node[buffer port] (E) {};
\node at (E.bout) [ocirc, right]{};
\draw (D.out) |- (B.in 1);
\draw (E.out) |- (A.in 2);
\draw (-7, 0.5) node (E1) {$E_0$};
\draw (E1) (-6.25,  0.5) node[circle, fill, inner sep=1pt] (E1cross) {} |- (D.in 1);
\draw (E1) to (-6.25, 0.5) |- (A.in 1);
\draw (-7, -0.75) node (E2) {$E_1$};
\draw (E2) (-6,  -0.75) node[circle, fill, inner sep=1pt] (E2cross) {} |- (E.in 1);
\draw (E2) to (-6, -0.75) |- (B.in 2);
\end{circuitikz}
\end{figure}
\end{solution}

\newpage

\begin{exercise}
\label{ex-xnor-schaltnetz}
Übersetzen Sie die Boolesche Formel $A_0 = E_0 \wedge E_1 \vee \neg E_0 \wedge \neg E_1$ in ein Schaltnetz.
\fillwithgrid{3in}
\end{exercise}

\begin{solution}
$A_0 = E_0 \wedge E_1 \vee \neg E_0 \wedge \neg E_1 = (E_0 \wedge E_1) \vee ((\neg E_0) \wedge (\neg E_1))$
\begin{figure}[htb]
\centering
\begin{circuitikz}
\draw (-2, 1) node[and port] (A) {};
\draw (-2, -1) node[and port] (B) {}; 
\draw (0,0) node[or port] (C) {}; 
\draw (A.out) |- (C.in 1);
\draw (B.out) |- (C.in 2);
\draw (C.out) node[anchor=west] {$A_0$};
\draw (-4,-0.5) node[buffer port] (D) {};
\node at (D.bout) [ocirc, right]{};
\draw (-4,-1.5) node[buffer port] (E) {};
\node at (E.bout) [ocirc, right]{};
\draw (D.out) |- (B.in 1);
\draw (E.out) |- (B.in 2);
\draw (-7, 0.5) node (E1) {$E_0$};
\draw (E1) (-6.25,  0.5) node[circle, fill, inner sep=1pt] (E1cross) {} |- (D.in 1);
\draw (E1) to (-6.25, 0.5) |- (A.in 1);
\draw (-7, -0.75) node (E2) {$E_1$};
\draw (E2) (-6,  -0.75) node[circle, fill, inner sep=1pt] (E2cross) {} |- (E.in 1);
\draw (E2) to (-6, -0.75) |- (A.in 2);
\end{circuitikz}
\end{figure}
\end{solution}

\begin{exercise}\label{exercise-xor-nor-nand-xnor}
Die Schaltnetze aus den letzten vier Übungen kommen sehr häufig vor. Deshalb ist es möglich, \textbf{das Schaltnetz} direkt als eigenes \textbf{Logikgatter} zu kaufen. \say{Hinter} dem Logikgatter \say{versteckt} sich quasi ein Schaltnetz mit \texttt{UND}-, \texttt{ODER}- und \texttt{NICHT}-Gattern. 

Für die Logikgatter werden die Bezeichnungen und Darstellungen aus \autoref{table-exercise-extended-logic-gates} benutzt. Füllen Sie \autoref{table-exercise-extended-logic-gates} wie folgt aus:

\begin{enumerate}
\item[a)] Tragen Sie für jedes Logikgatter das dazugehörige Schaltnetz ein. Sie müssen nicht das Schaltnetz zeichnen, sondern mit Übung \ref{ex-nand-schaltnetz}, Übung \ref{ex-nor-schaltnetz}, Übung \ref{ex-xor-schaltnetz} oder Übung \ref{ex-xnor-schaltnetz} auf das dazugehörige Schaltnetz aus den vorherigen Übungen verweisen.

\item[b)] Fügen Sie in die Spalte Beschreibung für jedes Logikgatter einen Satz hinzu, welcher erklärt, wann die \say{Glühbirne} leuchtet. Sie finden alle Logikgatter auch in Logicly.
\end{enumerate}

\begin{table}[htb]
\centering
\begin{tblr}{|Q[c, m, 2.7cm]|Q[c, m, 2.5cm]|c|Q[c, m, 3cm]|Q[c, m, 3cm]|}
\hline
\textbf{Bezeichnung} & \textbf{Darstellung} & \textbf{Schaltnetz} &  \textbf{Formel} & \textbf{Beschreibung} \\ \hline
{
\textbf{\texttt{XOR}} \\
(e\textbf{x}clusive \textbf{or}, \\ dt. entweder oder)
}
&
\begin{circuitikz}
\draw (0, 0) node[xor port] (A) {};
\end{circuitikz} \\ \hline
{
\textbf{\texttt{NAND}} \\
(\textbf{n}ot \textbf{and})
}
&
\begin{circuitikz}
\draw (0, 0) node[nand port] (A) {};
\end{circuitikz} \\ \hline
{
\textbf{\texttt{XNOR}} \\
(e\textbf{x}clusive \textbf{n}ot \textbf{or}, dt. Äquivalenz)
}
&
\begin{circuitikz}
\draw (0, 0) node[xnor port] (A) {};
\end{circuitikz} \\ \hline
{
\textbf{\texttt{NOR}} \\
(\textbf{n}ot \textbf{or})
}
&
\begin{circuitikz}
\draw (0, 0) node[nor port] (A) {};
\end{circuitikz} \\ \hline
\end{tblr}
\caption{Häufige Logikgatter, welche aus Grundgattern zusammengesetzt sind.}
\label{table-exercise-extended-logic-gates}
\end{table}
\end{exercise}

\newpage

\section{Ausfüllen von Wahrheitstabellen}

Wir zeigen nun, wie wir eine Wahrheitstabelle ausfüllen. Entweder nutzen wir die Boolesche Formel oder das Schaltnetz. Wir zeigen beide Varianten an einem Beispiel.

\begin{example}
\label{example-wahrheitstabelle-ausfuellen-bsp}
Wir zeigen, wie wir die Wahrheitstabelle für
\begin{align}
A_0 = (\neg E_0) \wedge E_1
\end{align}
ausfüllen. In \autoref{table-wahrheitstabelle-bsp-leer} ist die Vorlage abgebildet.

\begin{table}[htb]
\centering
\begin{tblr}{|c|c||c|}
\hline
$E_1$ & $E_0$ & $A_0 = \neg E_0 \wedge E_1$ \\ \hline[2pt]
$0$    	&  $0$     	&  ?	\\ \hline
$0$     	& $1$     	&  ?	\\ \hline
$1$ 		& $0$      	&  ? 	\\ \hline
$1$     	& $1$     	&  ?	\\ \hline
\end{tblr}
\caption{Die vier Werte für den Ausgang $A_0$ sind gesucht.}
\label{table-wahrheitstabelle-bsp-leer}
\end{table}

Wir haben zwei Eingänge, somit vier Zeilen. Für jede Belegung untersuchen wir, ob der Ausgang $0$ oder $1$ ergibt. Wir können dies Schritt-für-Schritt durchführen.
\end{example}

\begin{important}[Zeilenanzahl einer Wahrheitstabelle]
Eine Wahrheitstabelle mit \textbf{zwei Eingängen} besitzt \textbf{vier Zeilen}. Eine Wahrheitstabelle mit \textbf{drei Eingängen} besitzt \textbf{acht Zeilen}. Eine Wahrheitstabelle mit $n$ Eingängen besitzt $2^n$ Zeilen.
\end{important}

\subsection{Das Ausfüllen der Wahrheitstabelle mithilfe des Schaltnetzes}

Dazu notieren wir nacheinander jede Belegung in das Schaltnetz und lassen die Werte durch das Schaltnetz \say{fliessen}. Wir tragen dann den Wert am Ausgang in die Wahrheitstabelle ein.

\begin{example}
\label{example-wahrheitstabelle-ausfuellen-bsp-schaltnetz}

Wir setzen \autoref{example-wahrheitstabelle-ausfuellen-bsp} fort. Dazu füllen wir \autoref{table-wahrheitstabelle-bsp-leer} nun Schritt-für-Schritt aus.

\begin{figure}[htb]
\centering
\begin{minipage}{0.475\textwidth}
\centering
\begin{circuitikz}
\draw (0,1) node (Header) {\textbf{1. Belegung: $E_0 = 0$ und $E_1 = 0$}};
\draw (0,0) node[not port] (NOT1) {}
(NOT1.in 1) node[anchor=east] {$E_0$}; 
\draw (NOT1.out) -- ++(0.5,0) node[and port, anchor=in 1] (AND1) {};
\draw (NOT1.in 1) ++ (0, -1) node[anchor=east] {$E_1$} -- (0.5,-1) |- (AND1.in 2);
\draw (AND1.out) node[anchor=west] {$A_0$};
\node at (-1.75,0.25) {$0$};
\node at (-1.75,-1.25) {$0$};
\node at (0.15,0.25) {$1$};
\node at (0.75,0.25) {$1$};
\node at (0.75,-0.8) {$0$};
\node at (2.75,0.1) {$0$};
\end{circuitikz}
\caption*{Das \texttt{NICHT} macht aus der $0$ von $E_0$ eine $1$. Am \texttt{UND} liegt deshalb oben eine $1$ und unten eine $0$ an. Dies ergibt am Ausgang $0$.}
\end{minipage}
\hfill
\begin{minipage}{0.475\textwidth}
\centering
\begin{circuitikz}
\draw (0,1) node (Header) {\textbf{2. Belegung: $E_0 = 0$ und $E_1 = 1$}};
\draw (0,0) node[not port] (NOT1) {}
(NOT1.in 1) node[anchor=east] {$E_0$}; 
\draw (NOT1.out) -- ++(0.5,0) node[and port, anchor=in 1] (AND1) {};
\draw (NOT1.in 1) ++ (0, -1) node[anchor=east] {$E_1$} -- (0.5,-1) |- (AND1.in 2);
\draw (AND1.out) node[anchor=west] {$A_0$};
\node at (-1.75,0.25) {$0$};
\node at (-1.75,-1.25) {$1$};
\node at (0.15,0.25) {$1$};
\node at (0.75,0.25) {$1$};
\node at (0.75,-0.8) {$1$};
\node at (2.75,0.1) {$1$};
\end{circuitikz}
\caption*{Das \texttt{NICHT} macht aus der $0$ von Eingang $E_0$ eine $1$. Am \texttt{UND} liegt deshalb oben eine $1$ unten eine $1$ an. Dies ergibt am Ausgang $1$.}
\end{minipage}
\end{figure}

\begin{figure}[htb]
\centering
\begin{minipage}{0.475\textwidth}
\centering
\begin{circuitikz}
\draw (0,1) node (Header) {\textbf{3. Belegung: $E_0 = 1$ und $E_1 = 0$}};
\draw (0,0) node[not port] (NOT1) {}
(NOT1.in 1) node[anchor=east] {$E_0$}; 
\draw (NOT1.out) -- ++(0.5,0) node[and port, anchor=in 1] (AND1) {};
\draw (NOT1.in 1) ++ (0, -1) node[anchor=east] {$E_1$} -- (0.5,-1) |- (AND1.in 2);
\draw (AND1.out) node[anchor=west] {$A_0$};
\node at (-1.75,0.25) {$1$};
\node at (-1.75,-1.25) {$0$};
\node at (0.15,0.25) {$0$};
\node at (0.75,0.25) {$0$};
\node at (0.75,-0.8) {$0$};
\node at (2.75,0.1) {$0$};
\end{circuitikz}
\caption*{Das \texttt{NICHT} macht aus der $1$ von $E_0$ eine $0$. Am \texttt{UND} liegt deshalb oben eine $0$ und unten eine $0$ an. Dies ergibt am Ausgang $0$.}
\end{minipage}
\hfill
\begin{minipage}{0.475\textwidth}
\centering
\begin{circuitikz}
\draw (0,1) node (Header) {\textbf{4. Belegung: $E_0 = 1$ und $E_1 = 1$}};
\draw (0,0) node[not port] (NOT1) {}
(NOT1.in 1) node[anchor=east] {$E_0$}; 
\draw (NOT1.out) -- ++(0.5,0) node[and port, anchor=in 1] (AND1) {};
\draw (NOT1.in 1) ++ (0, -1) node[anchor=east] {$E_1$} -- (0.5,-1) |- (AND1.in 2);
\draw (AND1.out) node[anchor=west] {$A_0$};
\node at (-1.75,0.25) {$1$};
\node at (-1.75,-1.25) {$1$};
\node at (0.15,0.25) {$0$};
\node at (0.75,0.25) {$0$};
\node at (0.75,-0.8) {$1$};
\node at (2.75,0.1) {$0$};
\end{circuitikz}
\caption{Das \texttt{NICHT} macht aus der $1$ von $E_0$ eine $0$. Am \texttt{UND} liegt deshalb oben eine $0$ und unten eine $1$ an. Dies ergibt am Ausgang $1$.}
\end{minipage}
\end{figure}
\end{example}

\subsection{Das Ausfüllen der Wahrheitstabelle mithilfe der Booleschen Formel}

Wir setzen jede Belegung in die Formel ein und werten die Formel dann aus. Dies kennen Sie bereits aus der Mathematik, wie folgendes Beispiel zeigt.

\begin{example}
Wir können das Volumen einer Kugel mit der Formel $V = \frac{4}{3} \cdot \pi \cdot r^3$ berechnen. Möchten wir nun für den Radius $\qty{2}{\centi\metre}$ das Volumen berechnen, dann setzen wir den Wert in die Formel ein und rechnen das Ergebnis aus.
\begin{align}
r = \qty{2}{\centi\metre} \Rightarrow V = \frac{4}{3} \cdot \pi \cdot (\qty{2}{\centi\metre})^3 \approx \qty{33,51}{\cubic\centi\metre}
\end{align}
\end{example}

Das gleiche Prinzip wenden wir nun auf Boolesche Formeln an. Wir ersetzen die Eingänge in der Formel durch eine Belegung und rechnen das Ergebnis aus.

\begin{example}
Wir setzen \autoref{example-wahrheitstabelle-ausfuellen-bsp} fort und ermitteln für $A_0 = \neg E_0 \wedge E_1 = (\neg E_0) \wedge E_1$ die fehlenden Werte der Wahrheitstabelle.
\begin{itemize}
\item $E_0 = 0$ und $E_1 = 0 \Rightarrow A_0 = (\overbrace{\neg 0}^{\text{Def.~}\texttt{NICHT}}) \wedge 0 = (1) \wedge 0 = \overbrace{1 \wedge 0}^{\text{Def.~}\texttt{UND}} = 0$.
\item $E_0 = 0$ und $E_1 = 1 \Rightarrow A_0 = (\neg 0) \wedge 1 = (1) \wedge 1 = 1 \wedge 1 = 1$. 
\item $E_0 = 1$ und $E_1 = 0 \Rightarrow A_0 = (\neg 1) \wedge 0 = (0) \wedge 0 = 0 \wedge 0 = 0$. 
\item $E_0 = 1$ und $E_1 = 1 \Rightarrow A_0 = (\neg 1) \wedge 1 = (0) \wedge 1 = 0 \wedge 1 = 0$. 
\end{itemize}
\end{example}

Egal mit welcher Methode wir arbeiten, wir erhalten die gleichen Ergebnisse für die Ausgänge einer Wahrheitstabelle.

\begin{example}
Sowohl die Methode mit dem Schaltnetz, als auch die Methode mit der Booleschen Formel ergeben die Wahrheitstabelle aus \autoref{table-wahrheitstabelle-bsp-voll}.

\begin{table}[htb]
\centering
\begin{tblr}{|c|c||c|}
\hline
$E_1$ & $E_0$ & $A_0 = \neg E_0 \wedge E_1$ \\ \hline[2pt]
$0$    	&  $0$     	&  0	\\ \hline
$0$     	& $1$     	&  1	\\ \hline
$1$ 		& $0$      	&  0 	\\ \hline
$1$     	& $1$     	&  0	\\ \hline
\end{tblr}
\caption{Die vier Werte für den Ausgang $A_0$ wurden berechnet.}
\label{table-wahrheitstabelle-bsp-voll}
\end{table}
\end{example}

\subsection{Übungen}

Sie müssen die Wahrheitstabelle nicht mit Lineal zeichnen. Eine saubere Darstellung der Spalten und Zeilen ist ausreichend.

\begin{exercise}
Erstellen Sie für die Boolesche Formel $A_0 = E_0 \vee \neg E_1$ die dazugehörige Wahrheitstabelle.
\fillwithgrid{\stretch{1}}
\end{exercise}

\begin{solution}
$A_0 = E_0 \vee \neg E_1 = E_0 \vee (\neg E_1)$
\begin{table}[htb]
\centering
\begin{tblr}{|c|c||c|}
\hline
$E_0$ 	& 	$E_1$ 	& 	$A_0 = E_0 \vee (\neg E_1)$ 	\\ \hline[2pt]
$0$		&  	$0$     	& 	$1$    					\\ \hline
$0$		& 	$1$     	& 	$0$   					\\ \hline
$1$ 		& 	$0$      	& 	$1$   					\\ \hline
$1$		& 	$1$     	& 	$1$     					\\ \hline
\end{tblr}
\end{table}
\end{solution}

\newpage

\begin{exercise}
Erstellen Sie für die Boolesche Formel $A_0 = \neg (E_0 \wedge E_1)$ die dazugehörige Wahrheitstabelle.
\fillwithgrid{1.75in}
\end{exercise}

\begin{solution}
\begin{table}[htb]
\centering
\begin{tblr}{|c|c||c|}
\hline
$E_0$ 	& 	$E_1$ 	& 	$A_0 = \neg (E_0 \wedge E_1)$ 	\\ \hline[2pt]
$0$		&  	$0$     	& 	$1$    						\\ \hline
$0$		& 	$1$     	& 	$1$   						\\ \hline
$1$ 		& 	$0$      	& 	$1$   						\\ \hline
$1$		& 	$1$     	& 	$0$     						\\ \hline
\end{tblr}
\end{table}
\end{solution}

\begin{exercise}
Erstellen Sie für die Boolesche Formel $A_0 = \neg (E_0 \vee E_1)$ die dazugehörige Wahrheitstabelle.
\fillwithgrid{1.75in}
\end{exercise}

\begin{solution}
\begin{table}[htb]
\centering
\begin{tblr}{|c|c||c|}
\hline
$E_0$ 	& 	$E_1$ 	& 	$A_0 = \neg (E_0 \vee E_1)$ 	\\ \hline[2pt]
$0$		&  	$0$     	& 	$1$    						\\ \hline
$0$		& 	$1$     	& 	$0$   						\\ \hline
$1$ 		& 	$0$      	& 	$0$   						\\ \hline
$1$		& 	$1$     	& 	$0$     						\\ \hline
\end{tblr}
\end{table}
\end{solution}

\begin{exercise}
Erstellen Sie für die Boolesche Formel $A_0 = E_0 \wedge \neg E_1 \vee \neg E_0 \wedge E_1$ die dazugehörige Wahrheitstabelle.
\fillwithgrid{1.75in}
\end{exercise}

\begin{solution}
$A_0 = E_0 \wedge \neg E_1 \vee \neg E_0 \wedge E_1 = (E_0 \wedge (\neg E_1)) \vee ((\neg E_0) \wedge E_1)$
\begin{table}[htb]
\centering
\begin{tblr}{|c|c||c|}
\hline
$E_0$ 	& 	$E_1$ 	& 	$A_0 = E_0 \wedge \neg E_1 \vee \neg E_0 \wedge E_1$ 	\\ \hline[2pt]
$0$		&  	$0$     	& 	$0$	\\ \hline
$0$		& 	$1$     	& 	$1$	\\ \hline
$1$ 		& 	$0$      	& 	$1$	\\ \hline
$1$		& 	$1$     	& 	$0$	\\ \hline
\end{tblr}
\end{table}
\end{solution}

\begin{exercise}
Erstellen Sie für die Boolesche Formel$A_0 = E_0 \wedge E_1 \vee \neg E_0 \wedge \neg E_1$ die dazugehörige Wahrheitstabelle.
\fillwithgrid{1.75in}
\end{exercise}

\begin{solution}
$A_0 = E_0 \wedge E_1 \vee \neg E_0 \wedge \neg E_1 = (E_0 \wedge E_1) \vee ((\neg E_0) \wedge (\neg E_1))$
\begin{table}[htb]
\centering
\begin{tblr}{|c|c||c|}
\hline
$E_0$ 	& 	$E_1$ 	& 	$A_0 = E_0 \wedge E_1 \vee \neg E_0 \wedge \neg E_1$ 	\\ \hline[2pt]
$0$		&  	$0$     	& 	$1$	\\ \hline
$0$		& 	$1$     	& 	$0$	\\ \hline
$1$ 		& 	$0$      	& 	$0$	\\ \hline
$1$		& 	$1$     	& 	$1$	\\ \hline
\end{tblr}
\end{table}
\end{solution}
