% !TEX root = ../../../main.tex

\toggletrue{image}
\toggletrue{imagehover}
\chapterimage{boat_puzzle}
\chapterimagetitle{\uppercase{Boat Puzzle}}
\chapterimageurl{https://xkcd.com/2348/}
\chapterimagehover{'No, my cabbage moths have already started laying eggs in them! Send the trolley into the river!' 'No, the sailing wolf will steal the boat to rescue them!'.}

\chapter{Disjunktive Normalform}
\label{chapter-dnf}

Um sinnvolle Anwendungen mit einem Schaltnetz möglichst einfach realisieren zu können, müssen wir eine boolesche Formel für eine Wahrheitstabelle konstruieren. Die Lernziele sind:\\

\newcommand{\dnfLernziele}{
\protect\begin{minipage}{\textwidth}
\begin{todolist}
\item Sie erstellen für eine Wahrheitstabelle die Boolesche Formel in disjunktiver Normalform (\acs{DNF}) und das zugehörige Schaltnetz.
\end{todolist}
\end{minipage}
}

\lernziel{\autoref{chapter-dnf}, \nameref{chapter-dnf}}{\protect\dnfLernziele}

\dnfLernziele

\section{Konstruktion der disjunktiven Normalform}
\label{section-dnf}

Wir können für \textbf{jede} Wahrheitstabelle eine boolesche Formel nach den folgenden Regeln erstellen:

\begin{important}
Für \textbf{jeden Ausgang} $A$:
\begin{enumerate}
\item[1.)] Suche \textbf{alle Zeilen} mit \textbf{Ausgang} $\mathbf{1}$.
\item[2.)] Für jede Zeile aus 1.): Verknüpfe die \textbf{Eingänge} dieser Zeile mit \textbf{Konjunktionen} (\texttt{UND}-Operator). Wenn der \textbf{Eingang} $\mathbf{E}$ den \textbf{Wert} $\mathbf{1}$ hat, dann notiere $\mathbf{E}$. Andernfalls (der Eingang hat den Wert $\mathbf{0}$) notiere $\mathbf{\neg E}$.
\item[3.)] Verknüpfe die \textbf{Zeilen} aus 2.) mit \textbf{Disjunktionen} (\texttt{ODER}-Operator).
\end{enumerate}
\end{important}

Dieses \say{Rezept} führt zu booleschen Formeln der folgenden Form:
\begin{align*}
A _1 = \overbrace{(E_0 \wedge E_1 \wedge \cdots \wedge E_n)}^{\text{Konjuktionen}} \tikzmark{disjunktion1}\vee \overbrace{(E_0 \wedge \neg E_1 \wedge \cdots \wedge E_n)}^{\text{Konjuktionen}} \tikzmark{disjunktion2}\vee \cdots \tikzmark{disjunktion3}\vee \overbrace{(\neg E_0 \wedge E_1 \wedge \cdots \wedge E_n)}^{\text{Konjuktionen}}
\begin{tikzpicture}[overlay, remember picture, shorten <=1mm, font=\footnotesize]
\draw[latex-] ([xshift=1.0ex] pic cs:disjunktion1) -- ++ (0,-0.5) node[below] {\text{Disjunktion}};
\draw[latex-] ([xshift=1.0ex] pic cs:disjunktion2) -- ++ (0.5,-0.5) node[below] {};
\draw[latex-] ([xshift=1.0ex] pic cs:disjunktion3) -- ++ (0,-0.5) node[below] {\text{Disjunktion}};
\end{tikzpicture}
\end{align*}

\vspace{0.75cm}

Wenn eine boolesche Formel diese Struktur hat, handelt es sich um eine boolesche Formel in \textbf{disjunktiver Normalform} (\ac{DNF}).

\begin{hinweis}
Es gibt mehrere boolesche Formeln für eine Wahrheitstabelle. Die booleschen Formeln in \ac{DNF} können für jede Wahrheitstabelle verwendet werden. Es kann jedoch eine boolesche Formel geben, die kürzer ist als die boolesche Formel in \textbf{DNF}.
\end{hinweis}

Wir zeigen nun an einem Beispiel, wie wir eine boolesche Formeln in \ac{DNF} konstruieren.

\newpage

\begin{example}
Für die Wahrheitstabelle in \autoref{dnf-bsp1} konstruieren wir eine boolesche Formel in \ac{DNF}. Dazu markieren wir alle Ausgänge, die den Wert $1$ haben (siehe \autoref{dnf-bsp1-schritt-1}).

\begin{table}[htb]
\centering
\begin{minipage}{0.45\textwidth}
\centering
\begin{tblr}{|c|c||c|}
\hline
$E_1$ & $E_0$ & $A_0 = \text{???}$ \\ \hline[2pt]
$0$    	&  $0$     	&  1	\\ \hline
$0$     	& $1$     	&  0	\\ \hline
$1$ 		& $0$      	&  1 	\\ \hline
$1$     	& $1$     	&  0	\\ \hline
\end{tblr}
\caption{Wahrheitstabelle für $A_0$.}
\label{dnf-bsp1}
\end{minipage}
\begin{minipage}{0.45\textwidth}
\centering
\begin{tblr}{
colspec = {|c|c||c|c},
cell{2}{3} = {yellow!25},
cell{4}{3} = {yellow!25}
}
\cline{1-3}
$E_1$ & $E_0$ & $A_0 = \text{???}$ \\ \cline[2pt]{1-3}
$0$    	& $0$     	&  1 & \circled{1} \\ \cline{1-3}
$0$     	& $1$     	&  0	\\ \cline{1-3}
$1$ 		& $0$      	&  1 & \circled{3} \\ \cline{1-3}
$1$     	& $1$     	&  0	\\ \cline{1-3}
\end{tblr}
\caption{Schritt 1}
\label{dnf-bsp1-schritt-1}
\end{minipage}
\end{table}

Dann konstruieren wir für jede dieser Zeilen (siehe gelbe Markierungen) eine Teilformel mithilfe von \textbf{Negationen} und \textbf{Konjunktionen}:

\begin{multicols}{2}
\begin{itemize}
\item Zeile \circled{1} ergibt: $\neg E_0 \wedge \neg E_1$
\item Zeile \circled{3} ergibt: $\neg E_0 \wedge E_1$
\end{itemize}
\end{multicols}

Abschliessend setzen wir die Teilformel mithilfe von \textbf{Disjunktionen} für $A_0$ zusammen:
\begin{align*}
A_0 = (\neg E_0 \wedge \neg E_1) \vee (\neg E_0 \wedge E_1)
\end{align*}
\end{example}

Mit der booleschen Formel in \ac{DNF} können wir wieder das zugehörige Schaltnetz zeichnen.

\section{Übungen}

Erstellen Sie \textbf{pro Übung} für \textbf{jeden Ausgang} die \textbf{boolesche Formel} in \ac{DNF}. Stellen Sie \textbf{alle Schritte sauber} dar. Zeichnen Sie dann das \textbf{entsprechende Schaltnetz}.

\begin{exercise}
\begin{table}[htb]
\centering
\begin{minipage}{0.4\textwidth}
\centering
\begin{tblr}{
colspec = {|c|c||c|}
}
\hline
$E_1$ & $E_0$ & $A_0$ \\ \hline[2pt]
$0$  & $0$  & $0$  \\ \hline
$0$  & $1$  & $1$  \\ \hline
$1$  & $0$  & $1$  \\ \hline
$1$ & $1$  & $1$  \\ \hline
\end{tblr}
\caption*{Wahrheitstabelle}
\label{table-dnf-uebung-1}
\end{minipage}
\hfill
\begin{minipage}{0.55\textwidth}
Boolesche Formel in \ac{DNF}:
\fillwithgrid{1.25in}
\end{minipage}
\end{table}
Schaltnetz:
\fillwithgrid{\stretch{1}}
\end{exercise}

\newpage

\begin{important}
Die Schaltnetze für die beiden folgenden Aufgaben können sehr gross sein. Sie können auch Logikgatter mit mehr als zwei Eingängen verwenden (siehe \autoref{circuit-3-inputs}).
\end{important}

\begin{figure}[htb]
\centering
\begin{minipage}{0.21\textwidth}
\centering
\begin{circuitikz}
\draw (0,0) node[and port, number inputs=3, circuitikz/tripoles/european and port/height=1.5] (AND1) {}
(AND1.in 1) node[anchor=east] {$E_0$} 
(AND1.in 2) node[anchor=east] {$E_1$}
(AND1.in 3) node[anchor=east] {$E_2$}
(AND1.out) node[anchor=west] {$A_0$};
\end{circuitikz}
\end{minipage}
\hfill
\begin{minipage}{0.21\textwidth}
\centering
\begin{circuitikz}
\draw (0,0) node[and port, number inputs=3, circuitikz/tripoles/european and port/height=1.5] (AND1) {}
(AND1.in 1) node[anchor=east] {$E_0$} 
(AND1.in 2) node[anchor=east] {$E_1$}
(AND1.in 3) node[anchor=east] {$E_2$}
(AND1.out) node[anchor=west] {$A_0$};
\node at (AND1.bin 2) [notcirc, left]{};
\node at (AND1.bin 3) [notcirc, left]{};
\end{circuitikz}
\end{minipage}
\hfill
\begin{minipage}{0.21\textwidth}
\centering
\begin{circuitikz}
\draw (0,0) node[or port, number inputs=3, circuitikz/tripoles/european or port/height=1.5] (OR1) {}
(OR1.in 1) node[anchor=east] {$E_0$} 
(OR1.in 2) node[anchor=east] {$E_1$}
(OR1.in 3) node[anchor=east] {$E_2$}
(OR1.out) node[anchor=west] {$A_0$};
\node at (OR1.bin 1) [notcirc, left]{};
\end{circuitikz}
\end{minipage}
\hfill
\begin{minipage}{0.21\textwidth}
\centering
\begin{circuitikz}
\draw (0,0) node[or port, number inputs=3, circuitikz/tripoles/european or port/height=1.5] (OR1) {}
(OR1.in 1) node[anchor=east] {$E_0$} 
(OR1.in 2) node[anchor=east] {$E_1$}
(OR1.in 3) node[anchor=east] {$E_2$}
(OR1.out) node[anchor=west] {$A_0$};
\end{circuitikz}
\end{minipage}
\caption{Logikgatter mit drei Eingängen. Einzelne Eingänge können auch negiert werden.}
\label{circuit-3-inputs}
\end{figure}

\begin{exercise}
\begin{table}[htb]
\centering
\begin{minipage}{0.4\textwidth}
\centering
\begin{tblr}{
colspec = {|c|c|c||c|}
}
\hline
$E_2$ & $E_1$ & $E_0$ & $A_0$ \\ \hline[2pt]
$0$  & $0$  & $0$  & $0$  \\ \hline
$0$  & $0$  & $1$  & $0$  \\ \hline
$0$  & $1$  & $0$  & $0$  \\ \hline
$0$  & $1$  & $1$  & $1$  \\ \hline
$1$  & $0$  & $0$  & $0$  \\ \hline
$1$  & $0$  & $1$  & $1$  \\ \hline
$1$  & $1$  & $0$  & $1$  \\ \hline
$1$  & $1$  & $1$  & $1$  \\ \hline
\end{tblr}
\caption*{Wahrheitstabelle}
\label{table-dnf-uebung-2}
\end{minipage}
\hfill
\begin{minipage}{0.55\textwidth}
Boolesche Formel in \ac{DNF}:
\fillwithgrid{2.5in}
\end{minipage}
\end{table}
Schaltnetz:
\fillwithgrid{\stretch{1}}
\end{exercise}

\newpage

\begin{exercise}
\begin{table}[htb]
\centering
\begin{minipage}{0.4\textwidth}
\centering
\begin{tblr}{
colspec = {|c|c|c||c|}
}
\hline
$E_2$ & $E_1$ & $E_0$ & $A_0$ \\ \hline[2pt]
$0$  & $0$  & $0$  & $0$  \\ \hline
$0$  & $0$  & $1$  & $1$  \\ \hline
$0$  & $1$  & $0$  & $0$  \\ \hline
$0$  & $1$  & $1$  & $0$  \\ \hline
$1$  & $0$  & $0$  & $1$  \\ \hline
$1$  & $0$  & $1$  & $1$  \\ \hline
$1$  & $1$  & $0$  & $1$  \\ \hline
$1$  & $1$  & $1$  & $1$  \\ \hline
\end{tblr}
\caption*{Wahrheitstabelle}
\label{table-dnf-uebung-3}
\end{minipage}
\hfill
\begin{minipage}{0.55\textwidth}
Boolesche Formel in \ac{DNF}:
\fillwithgrid{2.5in}
\end{minipage}
\end{table}
Schaltnetz:
\fillwithgrid{\stretch{1}}
\end{exercise}

\begin{exercise}
\begin{table}[htb]
\centering
\begin{minipage}{0.2\textwidth}
\centering
\begin{tblr}{
colspec = {|c|c||c|}
}
\hline
$E_1$ & $E_0$ & $A_0$ \\ \hline[2pt]
$0$    &  $0$     & $1$    \\ \hline
$0$     & $1$     & $0$   \\ \hline
$1$ & $0$      & $0$   \\ \hline
$1$     & $1$     & $0$     \\ \hline
\end{tblr}
\caption*{Wahrheitstabelle}
\label{table-dnf-uebung-4}
\end{minipage}
\hfill
\begin{minipage}{0.75\textwidth}
Boolesche Formel in \ac{DNF} und Schaltnetz:
\fillwithgrid{1.5in}
\end{minipage}
\end{table}
\end{exercise}