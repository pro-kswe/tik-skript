% !TEX root = ../../../main.tex

\toggletrue{image}
\toggletrue{imagehover}
\chapterimage{boat_puzzle}
\chapterimagetitle{\uppercase{Boat Puzzle}}
\chapterimageurl{https://xkcd.com/2348/}
\chapterimagehover{'No, my cabbage moths have already started laying eggs in them! Send the trolley into the river!' 'No, the sailing wolf will steal the boat to rescue them!'.}

\chapter{Disjunktive Normalform}
\label{chapter-dnf}

Um sinnvolle Anwendungen mit einem Schaltnetz möglichst einfach realisieren zu können, müssen wir eine boolesche Formel für eine Wahrheitstabelle konstruieren. Die Lernziele sind:\\

\newcommand{\dnfLernziele}{
\protect\begin{minipage}{\textwidth}
\begin{todolist}
\item Sie erstellen für eine Wahrheitstabelle die Boolesche Formel in disjunktiver Normalform (\acs{DNF}) und das zugehörige Schaltnetz.
\end{todolist}
\end{minipage}
}

\lernziel{\autoref{chapter-dnf}, \nameref{chapter-dnf}}{\protect\dnfLernziele}

\dnfLernziele

\section{Konstruktion der disjunktiven Normalform}
\label{section-dnf}

Wir können für \textbf{jede} Wahrheitstabelle eine boolesche Formel nach den folgenden Regeln erstellen:

\begin{important}
Für \textbf{jeden Ausgang} $A$:
\begin{enumerate}
\item[1.)] Suche \textbf{alle Zeilen} mit \textbf{Ausgang $\mathbf{1}$}.
\item[2.)] Für jede Zeile aus 1.): Verknüpfe die \textbf{Eingänge} dieser Zeile mit \textbf{Konjunktionen} (\texttt{UND}-Operator). Wenn der \textbf{Eingang $\mathbf{E}$} den \textbf{Wert $\mathbf{1}$} hat, dann notiere $\mathbf{E}$. Andernfalls (der Eingang hat den Wert $\mathbf{0}$) notiere $\mathbf{\neg E}$.
\item[3.)] Verknüpfe die \textbf{Zeilen} aus 2.) mit \textbf{Disjunktionen} (\texttt{ODER}-Operator).
\end{enumerate}
\end{important}

Dieses \say{Rezept} führt zu booleschen Formeln der folgenden Form:
\begin{align*}
A _0 = \overbrace{(E_0 \wedge E_1 \wedge \cdots \wedge E_n)}^{\text{Konjuktionen}} \tikzmark{disjunktion1}\vee \overbrace{(E_0 \wedge \neg E_1 \wedge \cdots \wedge E_n)}^{\text{Konjuktionen}} \tikzmark{disjunktion2}\vee \cdots \tikzmark{disjunktion3}\vee \overbrace{(\neg E_0 \wedge E_1 \wedge \cdots \wedge E_n)}^{\text{Konjuktionen}}
\begin{tikzpicture}[overlay, remember picture, shorten <=1mm, font=\footnotesize]
\draw[latex-] ([xshift=1.0ex] pic cs:disjunktion1) -- ++ (0,-0.5) node[below] {\text{Disjunktion}};
\draw[latex-] ([xshift=1.0ex] pic cs:disjunktion2) -- ++ (0.5,-0.5) node[below] {};
\draw[latex-] ([xshift=1.0ex] pic cs:disjunktion3) -- ++ (0,-0.5) node[below] {\text{Disjunktion}};
\end{tikzpicture}
\end{align*}

\vspace{0.75cm}

Wenn eine boolesche Formel diese Struktur hat, handelt es sich um eine boolesche Formel in \textbf{disjunktiver Normalform} (\ac{DNF}).

\begin{hinweis}
Es gibt mehrere boolesche Formeln für eine Wahrheitstabelle. Die booleschen Formeln in \ac{DNF} können für jede Wahrheitstabelle verwendet werden. Es kann jedoch eine boolesche Formel geben, die kürzer ist als die boolesche Formel in \textbf{DNF}.
\end{hinweis}

Wir zeigen nun an einem Beispiel, wie wir eine boolesche Formeln in \ac{DNF} konstruieren.

\newpage

\begin{example}
Für die Wahrheitstabelle aus \autoref{dnf-bsp1} konstruieren wir eine boolesche Formel in \ac{DNF}. Dazu markieren wir alle Ausgänge, die den Wert $1$ haben (siehe \autoref{dnf-bsp1-schritt-1}).

\begin{table}[htb]
\centering
\begin{minipage}{0.45\textwidth}
\centering
\begin{tblr}{|c|c||c|}
\hline
$E_1$ & $E_0$ & $A_0 = \text{???}$ \\ \hline[2pt]
$0$    	&  $0$     	&  1	\\ \hline
$0$     	& $1$     	&  0	\\ \hline
$1$ 		& $0$      	&  1 	\\ \hline
$1$     	& $1$     	&  0	\\ \hline
\end{tblr}
\caption{Wahrheitstabelle für $A_0$.}
\label{dnf-bsp1}
\end{minipage}
\begin{minipage}{0.45\textwidth}
\centering
\begin{tblr}{
colspec = {|c|c||c|c},
cell{2}{3} = {yellow!25},
cell{4}{3} = {yellow!25}
}
\cline{1-3}
$E_1$ & $E_0$ & $A_0 = \text{???}$ \\ \cline[2pt]{1-3}
$0$    	& $0$     	&  1 & \circled{1} \\ \cline{1-3}
$0$     	& $1$     	&  0	\\ \cline{1-3}
$1$ 		& $0$      	&  1 & \circled{3} \\ \cline{1-3}
$1$     	& $1$     	&  0	\\ \cline{1-3}
\end{tblr}
\caption{Schritt 1}
\label{dnf-bsp1-schritt-1}
\end{minipage}
\end{table}

Dann konstruieren wir für jede dieser Zeilen (siehe gelbe Markierungen) eine Teilformel mithilfe von \textbf{Negationen} und \textbf{Konjunktionen}:

\begin{multicols}{2}
\begin{itemize}
\item Zeile \circled{1} ergibt: $\neg E_0 \wedge \neg E_1$
\item Zeile \circled{3} ergibt: $\neg E_0 \wedge E_1$
\end{itemize}
\end{multicols}

Abschliessend setzen wir die Teilformel mithilfe von \textbf{Disjunktionen} für $A_0$ zusammen:
\begin{align*}
A_0 = (\neg E_0 \wedge \neg E_1) \vee (\neg E_0 \wedge E_1)
\end{align*}
\end{example}

Mit der booleschen Formel in \ac{DNF} können wir wieder das zugehörige Schaltnetz zeichnen.

\section{Übungen}

Erstellen Sie \textbf{pro Übung} für \textbf{jeden Ausgang} die \textbf{boolesche Formel} in \ac{DNF}. Stellen Sie \textbf{alle Schritte sauber} dar. Zeichnen Sie dann das \textbf{entsprechende Schaltnetz}.

\begin{exercise}
\begin{figure}[H]
\centering
\begin{minipage}{0.4\textwidth}
\centering
\begin{tblr}{
colspec = {|c|c||c|c}
}
\hline
$E_1$ & $E_0$ & $A_0$ \\ \hline[2pt]
$0$  & $0$  & $0$  \\ \hline
$0$  & $1$  & $1$  \\ \hline
$1$  & $0$  & $1$  \\ \hline
$1$ & $1$  & $1$  \\ \hline
\end{tblr}
\caption*{Wahrheitstabelle}
\label{table-dnf-uebung-1}
\end{minipage}
\hfill
\begin{minipage}{0.55\textwidth}
Boolesche Formel in \ac{DNF}:
\fillwithgrid{1.25in}
\end{minipage}
\end{figure}
Schaltnetz:
\fillwithgrid{\stretch{1}}
\end{exercise}

\begin{solution}
\begin{figure}[H]
\begin{minipage}{0.4\textwidth}
\centering
\begin{tblr}{
colspec = {|c|c||c|},
cell{3}{3} = {yellow!25},
cell{4}{3} = {yellow!25},
cell{5}{3} = {yellow!25}
}
\hline
$E_1$ & $E_0$ & $A_0$ \\ \hline[2pt]
$0$  & $0$  & $0$  \\ \hline
$0$  & $1$  & $1$  \\ \hline
$1$  & $0$  & $1$  \\ \hline
$1$ & $1$  & $1$  \\ \hline
\end{tblr}
\caption*{Wahrheitstabelle}
\end{minipage}
\hfill
\begin{minipage}{0.55\textwidth}
\begin{flalign*}
\begin{rcases}
E_0 \wedge \neg E_1\\
\neg E_0 \wedge E_1\\
E_0 \wedge E_1
\end{rcases} & A_0 = (E_0 \wedge \neg E_1) \vee (\neg E_0 \wedge E_1) \vee (E_0 \wedge E_1) && 
\end{flalign*}
\caption*{Boolesche Formel in \ac{DNF}}
\end{minipage}
\vspace{0.5cm}
\begin{minipage}{0.6\textwidth}
\centering
\begin{circuitikz}
\draw (-2, 1.5) node[and port] (AND1) {};
\draw (-2, 0) node[and port] (AND2) {};
\draw (-2, -1.5) node[and port] (AND3) {};

\node at (AND1.bin 2) [notcirc, left]{};
\node at (AND2.bin 1) [notcirc, left]{};

\draw (0, 0.5) node[or port] (OR1) {};
\draw (2, 0) node[or port] (OR2) {};
\draw (OR2.out) node[anchor=west] {$A_0$};

\draw (AND1.out) |- (OR1.in 1);
\draw (AND2.out) |- (OR1.in 2);
\draw (OR1.out) |- (OR2.in 1);
\draw (AND3.out) |- (OR2.in 2);

\draw (-5, 1) node (E0) {$E_0$};
\draw (-5, -1) node (E1) {$E_1$};

\node[circle, fill, inner sep=1pt] (E0cross1) at (-4.25,  1) {};
\node[circle, fill, inner sep=1pt] (E0cross2) at (-4.25,  0.225) {};

\draw (E0)[xshift = 0.25cm] |- (E0cross1) |- (AND1.in 1);
\draw (E0cross1) |- (E0cross2);
\draw (E0cross1) |- (AND2.in 1);
\draw (E0cross2) |- (AND3.in 1);

\node[circle, fill, inner sep=1pt] (E1cross1) at (-4,  -1) {};
\node[circle, fill, inner sep=1pt] (E1cross2) at (-4,  -0.225) {};

\draw (E1)[xshift = 0.25cm] |- (E1cross1) |- (AND1.in 2);
\draw (E1cross1) |- (E1cross2);
\draw (E1cross2) |- (AND2.in 2);
\draw (E1cross1) |- (AND3.in 2);

\end{circuitikz}
\caption*{Schaltnetz}
\end{minipage}
\begin{minipage}{0.35\textwidth}
\centering
\begin{circuitikz}
\draw (-2, 1.5) node[and port] (AND1) {};
\draw (-2, 0) node[and port] (AND2) {};
\draw (-2, -1.5) node[and port] (AND3) {};

\node at (AND1.bin 2) [notcirc, left]{};
\node at (AND2.bin 1) [notcirc, left]{};

\draw (0, 0) node[or port, number inputs=3] (OR1) {};
\draw (OR1.out) node[anchor=west] {$A_0$};

\draw (AND1.out) |- (OR1.in 1);
\draw (AND2.out) |- (OR1.in 2);
\draw (AND3.out) |- (OR1.in 3);

\draw (-5, 1) node (E0) {$E_0$};
\draw (-5, -1) node (E1) {$E_1$};

\node[circle, fill, inner sep=1pt] (E0cross1) at (-4.25,  1) {};
\node[circle, fill, inner sep=1pt] (E0cross2) at (-4.25,  0.225) {};

\draw (E0)[xshift = 0.25cm] |- (E0cross1) |- (AND1.in 1);
\draw (E0cross1) |- (E0cross2);
\draw (E0cross1) |- (AND2.in 1);
\draw (E0cross2) |- (AND3.in 1);

\node[circle, fill, inner sep=1pt] (E1cross1) at (-4,  -1) {};
\node[circle, fill, inner sep=1pt] (E1cross2) at (-4,  -0.225) {};

\draw (E1)[xshift = 0.25cm] |- (E1cross1) |- (AND1.in 2);
\draw (E1cross1) |- (E1cross2);
\draw (E1cross2) |- (AND2.in 2);
\draw (E1cross1) |- (AND3.in 2);
\end{circuitikz}
\caption*{Alternatives Schaltnetz mit einem \texttt{ODER}-Gatter mit drei Eingängen.}
\end{minipage}
\end{figure}
\end{solution}

\newpage

\begin{important}
Die Schaltnetze für die beiden folgenden Aufgaben können sehr gross sein. Sie können auch Logikgatter mit mehr als zwei Eingängen verwenden (siehe \autoref{circuit-3-inputs}).
\end{important}

\begin{figure}[htb]
\centering
\begin{minipage}{0.21\textwidth}
\centering
\begin{circuitikz}
\draw (0,0) node[and port, number inputs=3, circuitikz/tripoles/european and port/height=1.5] (AND1) {}
(AND1.in 1) node[anchor=east] {$E_0$} 
(AND1.in 2) node[anchor=east] {$E_1$}
(AND1.in 3) node[anchor=east] {$E_2$}
(AND1.out) node[anchor=west] {$A_0$};
\end{circuitikz}
\end{minipage}
\hfill
\begin{minipage}{0.21\textwidth}
\centering
\begin{circuitikz}
\draw (0,0) node[and port, number inputs=3, circuitikz/tripoles/european and port/height=1.5] (AND1) {}
(AND1.in 1) node[anchor=east] {$E_0$} 
(AND1.in 2) node[anchor=east] {$E_1$}
(AND1.in 3) node[anchor=east] {$E_2$}
(AND1.out) node[anchor=west] {$A_0$};
\node at (AND1.bin 2) [notcirc, left]{};
\node at (AND1.bin 3) [notcirc, left]{};
\end{circuitikz}
\end{minipage}
\hfill
\begin{minipage}{0.21\textwidth}
\centering
\begin{circuitikz}
\draw (0,0) node[or port, number inputs=3, circuitikz/tripoles/european or port/height=1.5] (OR1) {}
(OR1.in 1) node[anchor=east] {$E_0$} 
(OR1.in 2) node[anchor=east] {$E_1$}
(OR1.in 3) node[anchor=east] {$E_2$}
(OR1.out) node[anchor=west] {$A_0$};
\node at (OR1.bin 1) [notcirc, left]{};
\end{circuitikz}
\end{minipage}
\hfill
\begin{minipage}{0.21\textwidth}
\centering
\begin{circuitikz}
\draw (0,0) node[or port, number inputs=3, circuitikz/tripoles/european or port/height=1.5] (OR1) {}
(OR1.in 1) node[anchor=east] {$E_0$} 
(OR1.in 2) node[anchor=east] {$E_1$}
(OR1.in 3) node[anchor=east] {$E_2$}
(OR1.out) node[anchor=west] {$A_0$};
\end{circuitikz}
\end{minipage}
\caption{Logikgatter mit drei Eingängen. Einzelne Eingänge können auch negiert werden.}
\label{circuit-3-inputs}
\end{figure}

\begin{exercise}
\begin{figure}[H]
\centering
\begin{minipage}{0.4\textwidth}
\centering
\begin{tblr}{
colspec = {|c|c|c||c|}
}
\hline
$E_2$ & $E_1$ & $E_0$ & $A_0$ \\ \hline[2pt]
$0$  & $0$  & $0$  & $0$  \\ \hline
$0$  & $0$  & $1$  & $0$  \\ \hline
$0$  & $1$  & $0$  & $0$  \\ \hline
$0$  & $1$  & $1$  & $1$  \\ \hline
$1$  & $0$  & $0$  & $0$  \\ \hline
$1$  & $0$  & $1$  & $1$  \\ \hline
$1$  & $1$  & $0$  & $1$  \\ \hline
$1$  & $1$  & $1$  & $1$  \\ \hline
\end{tblr}
\caption*{Wahrheitstabelle}
\label{table-dnf-uebung-2}
\end{minipage}
\hfill
\begin{minipage}{0.55\textwidth}
Boolesche Formel in \ac{DNF}:
\fillwithgrid{2.5in}
\end{minipage}
\end{figure}
Schaltnetz:
\fillwithgrid{\stretch{1}}
\end{exercise}

\begin{solution}
\begin{figure}[H]
\centering
\begin{minipage}{0.3\textwidth}
\centering
\begin{tblr}{
colspec = {|c|c|c||c|},
cell{5}{4} = {yellow!25},
cell{7}{4} = {yellow!25},
cell{8}{4} = {yellow!25},
cell{9}{4} = {yellow!25}
}
\hline
$E_2$ & $E_1$ & $E_0$ & $A_0$ \\ \hline[2pt]
$0$  & $0$  & $0$  & $0$  \\ \hline
$0$  & $0$  & $1$  & $0$  \\ \hline
$0$  & $1$  & $0$  & $0$  \\ \hline
$0$  & $1$  & $1$  & $1$  \\ \hline
$1$  & $0$  & $0$  & $0$  \\ \hline
$1$  & $0$  & $1$  & $1$  \\ \hline
$1$  & $1$  & $0$  & $1$  \\ \hline
$1$  & $1$  & $1$  & $1$  \\ \hline
\end{tblr}
\caption*{Wahrheitstabelle}
\end{minipage}
\hfill
\begin{minipage}{0.65\textwidth}
\centering
\begin{circuitikz}
\draw (-2, 2) node[and port, number inputs=3] (AND1) {};
\draw (-2, 0.75) node[and port, number inputs=3] (AND2) {};
\draw (-2, -0.75) node[and port, number inputs=3] (AND3) {};
\draw (-2, -2) node[and port, number inputs=3] (AND4) {};

\draw (1, 0) node[or port, number inputs=4] (OR1) {};
\draw (OR1.out) node[anchor=west] {$A_0$};

\node (AND1outCross) at (AND1.out)[xshift = 0.5cm] {};
\draw (AND1.out) |- (AND1outCross.center) |- (OR1.in 1);
\draw (AND2.out) |- (OR1.in 2);
\draw (AND3.out) |- (OR1.in 3);
\node (AND4outCross) at (AND4.out)[xshift = 0.5cm] {};
\draw (AND4.out) |- (AND4outCross.center) |- (OR1.in 4);

\draw (-7, 1.5) node (E0) {$E_0$};
\draw (-7, 0) node (E1) {$E_1$};
\draw (-7, -1.5) node (E2) {$E_2$};

\node[circle, fill, inner sep=1pt] (E0cross1) at (-6,  1.5) {};
\node[circle, fill, inner sep=1pt] (E0cross2) at (-6,  1.05) {};
\node[circle, fill, inner sep=1pt] (E0cross3) at (-6,  -0.45) {};
\draw (E0)[xshift = 0.25cm] |- (E0cross1);
\draw (E0cross1) |- (AND1.bin 1);
\draw (E0cross1) |- (E0cross2);
\draw (E0cross2) |- (AND2.bin 1);
\draw (E0cross2) |- (E0cross3);
\draw (E0cross3) |- (AND3.bin 1);
\draw (E0cross3) |- (AND4.bin 1);

\node[circle, fill, inner sep=1pt] (E1cross1) at (-5.5,  0) {};
\node[circle, fill, inner sep=1pt] (E1cross2) at (-5.5,  0.75) {};
\node[circle, fill, inner sep=1pt] (E1cross3) at (-5.5,  -0.75) {};
\draw (E1)[xshift = 0.25cm] |- (E1cross1);
\draw (E1cross1) |- (E1cross2);
\draw (E1cross2) |- (AND1.bin 2);
\draw (E1cross2) |- (AND2.bin 2);
\draw (E1cross1) |- (E1cross3);
\draw (E1cross3) |- (AND3.bin 2);
\draw (E1cross3) |- (AND4.bin 2);

\node[circle, fill, inner sep=1pt] (E2cross1) at (-5,  -1.5) {};
\node[circle, fill, inner sep=1pt] (E2cross2) at (-5,  -1.05) {};
\node[circle, fill, inner sep=1pt] (E2cross3) at (-5,  0.45) {};
\draw (E2)[xshift = 0.25cm] |- (E2cross1);
\draw (E2cross1) |- (E2cross2);
\draw (E2cross1) |- (AND4.bin 3);
\draw (E2cross2) |- (AND3.bin 3);
\draw (E2cross2) |- (E2cross3);
\draw (E2cross3) |- (AND1.bin 3);
\draw (E2cross3) |- (AND2.bin 3);

\node at (AND1.bin 3) [notcirc, left]{};
\node at (AND2.bin 2) [notcirc, left]{};
\node at (AND3.bin 1) [notcirc, left]{};

\end{circuitikz}
\caption*{Schaltnetz}
\end{minipage}
\vspace{0.5cm}
\begin{minipage}{\textwidth}
\centering
\begin{align*}
\begin{rcases}
E_0 \wedge E_1 \wedge \neg E_2 \\
E_0 \wedge \neg E_1 \wedge E_2 \\
\neg E_0 \wedge E_1 \wedge E_2 \\
E_0 \wedge E_1 \wedge E_2 \\
\end{rcases} & A_0 = (E_0 \wedge E_1 \wedge \neg E_2) \vee (E_0 \wedge \neg E_1 \wedge E_2) \vee (\neg E_0 \wedge E_1 \wedge E_2) \vee (E_0 \wedge E_1 \wedge E_2)
\end{align*}
\caption*{Boolesche Formel in \ac{DNF}}
\end{minipage}
\end{figure}
\end{solution}

\newpage

\begin{exercise}
\begin{table}[H]
\centering
\begin{minipage}{0.4\textwidth}
\centering
\begin{tblr}{
colspec = {|c|c|c||c|}
}
\hline
$E_2$ & $E_1$ & $E_0$ & $A_0$ \\ \hline[2pt]
$0$  & $0$  & $0$  & $0$  \\ \hline
$0$  & $0$  & $1$  & $1$  \\ \hline
$0$  & $1$  & $0$  & $0$  \\ \hline
$0$  & $1$  & $1$  & $0$  \\ \hline
$1$  & $0$  & $0$  & $1$  \\ \hline
$1$  & $0$  & $1$  & $1$  \\ \hline
$1$  & $1$  & $0$  & $1$  \\ \hline
$1$  & $1$  & $1$  & $1$  \\ \hline
\end{tblr}
\caption*{Wahrheitstabelle}
\label{table-dnf-uebung-3}
\end{minipage}
\hfill
\begin{minipage}{0.55\textwidth}
Boolesche Formel in \ac{DNF}:
\fillwithgrid{2.5in}
\end{minipage}
\end{table}
Schaltnetz:
\fillwithgrid{\stretch{1}}
\end{exercise}

\begin{solution}
\begin{figure}[H]
\centering
\begin{minipage}{0.3\textwidth}
\centering
\begin{tblr}{
colspec = {|c|c|c||c|},
cell{3}{4} = {yellow!25},
cell{6}{4} = {yellow!25},
cell{7}{4} = {yellow!25},
cell{8}{4} = {yellow!25},
cell{9}{4} = {yellow!25}
}
\hline
$E_2$ & $E_1$ & $E_0$ & $A_0$ \\ \hline[2pt]
$0$  & $0$  & $0$  & $0$  \\ \hline
$0$  & $0$  & $1$  & $1$  \\ \hline
$0$  & $1$  & $0$  & $0$  \\ \hline
$0$  & $1$  & $1$  & $0$  \\ \hline
$1$  & $0$  & $0$  & $1$  \\ \hline
$1$  & $0$  & $1$  & $1$  \\ \hline
$1$  & $1$  & $0$  & $1$  \\ \hline
$1$  & $1$  & $1$  & $1$  \\ \hline
\end{tblr}
\caption*{Wahrheitstabelle}
\end{minipage}
\hfill
\begin{minipage}{0.65\textwidth}
\centering
\begin{circuitikz}
\draw (-2, 2) node[and port, number inputs=3] (AND1) {};
\draw (-2, 1) node[and port, number inputs=3] (AND2) {};
\draw (-2, 0) node[and port, number inputs=3] (AND3) {};
\draw (-2, -1) node[and port, number inputs=3] (AND4) {};
\draw (-2, -2) node[and port, number inputs=3] (AND5) {};

\draw (1, 0) node[or port, number inputs=5] (OR1) {};
\draw (OR1.out) node[anchor=west] {$A_0$};

\node (AND1outCross) at (AND1.out)[xshift = 0.5cm] {};
\draw (AND1.out) |- (AND1outCross.center) |- (OR1.in 1);
\draw (AND2.out) |- (OR1.in 2);
\draw (AND3.out) |- (OR1.in 3);
\node (AND4outCross) at (AND4.out)[xshift = 0.5cm] {};
\draw (AND4.out) |- (AND4outCross.center) |- (OR1.in 4);
\node (AND5outCross) at (AND5.out)[xshift = 0.75cm] {};
\draw (AND5.out) |- (AND5outCross.center) |- (OR1.in 5);

\draw (-7, 1.5) node (E0) {$E_0$};
\draw (-7, 0) node (E1) {$E_1$};
\draw (-7, -1.5) node (E2) {$E_2$};

\node[circle, fill, inner sep=1pt] (E0cross1) at (-6,  1.5) {};
\node[circle, fill, inner sep=1pt] (E0cross2) at (-6,  1.305) {};
\node[circle, fill, inner sep=1pt] (E0cross3) at (-6,  0.3) {};
\node[circle, fill, inner sep=1pt] (E0cross4) at (-6,  -0.705) {};
\draw (E0)[xshift = 0.25cm] |- (E0cross1);
\draw (E0cross1) |- (AND1.bin 1);
\draw (E0cross1) |- (E0cross2);
\draw (E0cross2) |- (AND2.bin 1);
\draw (E0cross2) |- (E0cross3);
\draw (E0cross3) |- (AND3.bin 1);
\draw (E0cross3) |- (E0cross4);
\draw (E0cross4) |- (AND4.bin 1);
\draw (E0cross4) |- (AND5.bin 1);

\node[circle, fill, inner sep=1pt] (E1cross1) at (-5.5,  0) {};
\node[circle, fill, inner sep=1pt] (E1cross2) at (-5.5,  1) {};
\node[circle, fill, inner sep=1pt] (E1cross3) at (-5.5,  -1) {};
\draw (E1)[xshift = 0.25cm] |- (E1cross1);
\draw (E1cross1) |- (AND3.bin 2);
\draw (E1cross1) |- (E1cross2);
\draw (E1cross2) |- (AND1.bin 2);
\draw (E1cross2) |- (AND2.bin 2);
\draw (E1cross2) |- (E1cross3);
\draw (E1cross3) |- (AND4.bin 2);
\draw (E1cross3) |- (AND5.bin 2);

\node[circle, fill, inner sep=1pt] (E2cross1) at (-5,  -1.5) {};
\node[circle, fill, inner sep=1pt] (E2cross2) at (-5,  -1.305) {};
\node[circle, fill, inner sep=1pt] (E2cross3) at (-5,  -0.305) {};
\node[circle, fill, inner sep=1pt] (E2cross4) at (-5,  0.705) {};
\draw (E2)[xshift = 0.25cm] |- (E2cross1);
\draw (E2cross1) |- (AND5.bin 3);
\draw (E2cross1) |- (E2cross2);
\draw (E2cross2) |- (AND4.bin 3);
\draw (E2cross2) |- (E2cross3);
\draw (E2cross3) |- (AND3.bin 3);
\draw (E2cross3) |- (E2cross4);
\draw (E2cross4) |- (AND2.bin 3);
\draw (E2cross4) |- (AND1.bin 3);

\node at (AND1.bin 2) [notcirc, left]{};
\node at (AND1.bin 3) [notcirc, left]{};

\node at (AND2.bin 1) [notcirc, left]{};
\node at (AND2.bin 2) [notcirc, left]{};

\node at (AND3.bin 2) [notcirc, left]{};

\node at (AND4.bin 1) [notcirc, left]{};

\end{circuitikz}
\caption*{Schaltnetz}
\end{minipage}
\vspace{0.5cm}
\begin{minipage}{\textwidth}
\centering
\begin{align*}
\begin{rcases}
E_0 \wedge \neg E_1 \wedge \neg E_2 \\
\neg E_0 \wedge \neg E_1 \wedge E_2 \\
E_0 \wedge \neg E_1 \wedge E_2 \\
\neg E_0 \wedge E_1 \wedge E_2 \\
E_0 \wedge E_1 \wedge E_2 \\
\end{rcases} & \parbox[c]{10cm}{$A_0 = (E_0 \wedge \neg E_1 \wedge \neg E_2) \vee (\neg E_0 \wedge \neg E_1 \wedge E_2) \vee (E_0 \wedge \neg E_1 \wedge E_2) \vee (\neg E_0 \wedge E_1 \wedge E_2) \vee (E_0 \wedge E_1 \wedge E_2)$}
\end{align*}
\caption*{Boolesche Formel in \ac{DNF}}
\end{minipage}
\end{figure}
\end{solution}


\begin{exercise}
\begin{table}[H]
\centering
\begin{minipage}{0.2\textwidth}
\centering
\begin{tblr}{
colspec = {|c|c||c|}
}
\hline
$E_1$ & $E_0$ & $A_0$ \\ \hline[2pt]
$0$    &  $0$     & $1$    \\ \hline
$0$     & $1$     & $0$   \\ \hline
$1$ & $0$      & $0$   \\ \hline
$1$     & $1$     & $0$     \\ \hline
\end{tblr}
\caption*{Wahrheitstabelle}
\label{table-dnf-uebung-4}
\end{minipage}
\hfill
\begin{minipage}{0.75\textwidth}
Boolesche Formel in \ac{DNF} und Schaltnetz:
\fillwithgrid{1.5in}
\end{minipage}
\end{table}
\end{exercise}

\begin{solution}
\begin{figure}[H]
\begin{minipage}{0.3\textwidth}
\centering
\begin{tblr}{
colspec = {|c|c||c|},
cell{2}{3} = {yellow!25}
}
\hline
$E_1$ & $E_0$ & $A_0$ \\ \hline[2pt]
$0$  & $0$  & $1$  \\ \hline
$0$  & $1$  & $0$  \\ \hline
$1$  & $0$  & $0$  \\ \hline
$1$ & $1$  & $0$  \\ \hline
\end{tblr}
\caption*{Wahrheitstabelle}
\end{minipage}
\hfill
\begin{minipage}{0.3\textwidth}
\centering
$A_0 = \neg E_0 \wedge \neg E_1$
\caption*{Boolesche Formel in \ac{DNF}}
\end{minipage}
\hfill
\begin{minipage}{0.3\textwidth}
\centering
\begin{circuitikz}
\draw (-2, 0) node[and port] (AND1) {};

\node at (AND1.bin 1) [notcirc, left]{};
\node at (AND1.bin 2) [notcirc, left]{};

\draw (AND1.out) node[anchor=west] {$A_0$};

\draw (-4.5, 0.25) node (E0) {$E_0$};
\draw (-4.5, -0.25) node (E1) {$E_1$};

\draw (E0)[xshift = 0.25cm] |- (AND1.in 1);
\draw (E1)[xshift = 0.25cm] |- (AND1.in 2);
\end{circuitikz}
\caption*{Schaltnetz}
\end{minipage}
\end{figure}
\end{solution}