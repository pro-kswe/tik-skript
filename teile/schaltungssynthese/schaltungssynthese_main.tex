% !TEX root = ../../main.tex

\part{Schaltungssynthese}
\label{part-schaltungssynthese}

\togglefalse{image}
\togglefalse{imagehover}

\chapter{Majorität}
\label{chapter-majoritaet}

\begin{exercise}
Es soll ein Schaltnetz mit \textbf{drei Eingängen} und \textbf{zwei Ausgängen} konstruiert werden. Der \textbf{erste Ausgang} ist nur dann an, falls \textbf{mehr} Eingänge an sind als aus. Der \textbf{zweite Ausgang} ist nur dann an, wenn \textbf{alle} Eingänge an sind.

\begin{enumerate}
\item[1.)] Erstellen Sie die Wahrheitstabelle für das beschriebene Szenario.
\item[2.)] Erstellen Sie für die Wahrheitstabelle aus 1.) eine möglichst einfache Boolesche Formel in \ac{DNF} (\ac{KV}-Diagramm verwenden).
\item[3.)] Zeichnen Sie das Schaltnetz zur Booleschen Formel aus 2.)
\item[4.)] Übertragen Sie das Schaltnetz in Logicly. Testen Sie das Schaltnetz ausführlich.
\end{enumerate}
\end{exercise}

\fillwithgrid{\stretch{1}}

\newpage

\chapter{Parität}
\label{chapter-paritaet}

\begin{exercise}
Es soll ein Schaltnetz mit \textbf{drei Eingängen} und \textbf{zwei Ausgängen} konstruiert werden. Der \textbf{erste Ausgang} ist nur dann an, falls die \textbf{Anzahl} der eingeschalteten Eingänge \textbf{gerade} ist. Der \textbf{zweite Ausgang} ist nur dann an, wenn die \textbf{Anzahl} der eingeschalteten Eingänge \textbf{ungerade} ist.

\begin{enumerate}
\item[1.)] Erstellen Sie die Wahrheitstabelle für das beschriebene Szenario.
\item[2.)] Erstellen Sie für die Wahrheitstabelle aus 1.) eine möglichst einfache Boolesche Formel in \ac{DNF} (\ac{KV}-Diagramm verwenden).
\item[3.)] Zeichnen Sie das Schaltnetz zur Booleschen Formel aus 2.)
\item[4.)] Übertragen Sie das Schaltnetz in Logicly. Testen Sie das Schaltnetz ausführlich.
\end{enumerate}
\end{exercise}

\fillwithgrid{\stretch{1}}

\newpage